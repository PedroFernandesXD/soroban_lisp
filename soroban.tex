\documentclass[12pt]{book}
\usepackage[brazil]{babel}
\usepackage[utf8x]{inputenc}
\usepackage{keystroke}
\usepackage{listings}
\usepackage[compact]{titlesec}
\usepackage{lipsum}
\usepackage[margin=2.5cm]{geometry}
\usepackage{mathtools}
\usepackage[T1]{fontenc}
\usepackage{upquote}
\usepackage{tcolorbox}
\usepackage{wrapfig}
\usepackage{pgf-soroban}
\usepackage{tikz}
\usepackage{enumerate}
\usepackage{circuitikz}
\usepackage{pdfpages}
\usepgflibrary{decorations.markings}


\titleformat{\chapter}[display]
            {\bfseries\normalsize}
            {\filright\chaptertitlename~\large\thechapter}
            {5pt}{\large}
            \titlespacing{\chapter}{0pt}{0pt}{4pt}


\title{Soroban}
\author{Ed}
\date{}

\newcounter{observershipCounter}
\setcounter{observershipCounter}{0}


\renewcommand{\colbil}{white}
\renewcommand{\coltig}{black}

\begin{document}
%% \maketitle

%%\thispagestyle{empty}

\chapter{Circuits}

\begin{circuitikz}\draw
(6,0) node[or port, rotate=180] (or0) {}
($(or0.in 2) + (1,0)$) node[and port, rotate=180] (and1) {}
(or0.in 2) to (and1.out)
($(and1.in 2) + (1,0)$) node (x2) {X}
($(and1.in 1) + (1,-1)$) node[not port, rotate=180] (not2) {}
($(not2.in) + (1,0)$) node (y3) {Y}
($(or0.in 1) + (1,-1)$) node[and port, rotate=180] (and1) {}
(or0.in 1) to (and1.out)
($(and1.in 2) + (1,0)$) node[not port, rotate=180] (not2) {}
($(not2.in) + (1,0)$) node (x3) {X}
($(and1.in 1) + (1,-0)$) node (y2) {Y}
;
\end{circuitikz}

\chapter{Soroban}
As colunas ({\em keta} em japonês) 
por onde deslizam as pedras
indicam centavos, décimos, unidades,
dezenas, centenas, milhares, etc. Pedra
abaixo da barra divisória (hari) chama-se
ichidama e vale
1 na casa das unidades, 10 nas dezenas,
100 nas centenas, etc. Pedra acima da hari
chama-se godama e vale 5 nas unidades,
50 nas dezenas, 500 nas centenas, etc.
Para zerar, coloca-se
o ábaco na vertical, de modo
que as ichidamas
afastem-se da barra divisória. Em seguida, deslizamos
o dedo polegar por baixo de
cada godama, separando-a da hari. 


\paragraph{75+82}= 157.
O ábaco está zerado quando
todas as pedras estão
afastadas da hari           .
Para efetuar essa soma,
marca-se 75 no ábaco. As unidades estão
na terceira coluna da direita para a esquerda,
onde há um ponto decimal na hari.
As dezenas estão na quarta coluna.
Para indicar 70 dezenas, seguramos a
pedra de 50 com o indicador e duas pedras
de 10 com o polegar. Um movimento de
pinça dos referidos dedos aproxima
20 e 50 da barra de separação. Em seguida,
puxamos a pedra de 5 para a barra de separação
com o indicador.

\ladj{0.25}
\begin{minipage}{0.4\textwidth}
\begin{tikzpicture}
   \tige{1}{0}{0}
   \tige{2}{0}{0}       %<- here a dot: unit million
   \tige{3}{0}{1}       %<- here no dot: hundreds of thousands
   \tige{4}{0}{0}       %<- here no dot: tens of thousands
   \tige{5}{0}{0}       %<- here a dot: units thousands
   \tige{6}{0}{1}       %% milhar
   \tige{7}{0}{0}       %% centena
   \tige{8}{0}{0}       %% dezena
   \tige{9}{0}{1}       %% unidade
   \tige{10}{0}{0}
   \tige{11}{0}{0}
   \cadre{11}
\end{tikzpicture}
\end{minipage}%
%
\begin{minipage}{0.5\textwidth}
\begin{tikzpicture}
   \tige{1}{0}{0}
   \tige{2}{0}{0}       %<- here a dot: unit million
   \tige{3}{0}{1}       %<- here no dot: hundreds of thousands
   \tige{4}{0}{0}       %<- here no dot: tens of thousands
   \tige{5}{0}{0}       %<- here a dot: units thousands
   \tige{6}{0}{1}
   \tige{7}{0}{0}
   \tige{8}{7}{0}       %<- here a dot: units
   \binoire{8}{10}{gray}
   \binoire{8}{5}{gray}
   \binoire{8}{4}{gray}
   \tige{9}{5}{1}
   \binoire{9}{10}{gray}
   \tige{10}{0}{0}
   \tige{11}{0}{0}
   \cadre{11}
\end{tikzpicture}
\end{minipage}

Para somar 82 a 75, deveríamos
começar por adicionar 80 à coluna das dezenas,
mas isso não é possível,
pois não temos 5+3 pedras disponíveis
na quarta hari da direita para a esquerda.
Então, acrescentamos
100 e tiramos 20. As pedras cortadas
representam 20 unidades subtraídas.
Agora, acrescentamos 2 unidades e
terminamos o cálculo de 75+82.\\

\begin{minipage}{0.4\textwidth}
\begin{tikzpicture}
   \tige{1}{0}{0}
   \tige{2}{0}{0}       %<- here a dot: unit million
   \tige{3}{0}{1}       %<- here no dot: hundreds of thousands
   \tige{4}{0}{0}       %<- here no dot: tens of thousands
   \tige{5}{0}{0}       %<- here a dot: units thousands
   \tige{6}{0}{1}
   \tige{7}{1}{0}
   \binoire{7}{5}{gray}
   \tige{8}{5}{0}       %<- here a dot: units
   \binoire{8}{10}{gray}
   \barbil{8}{4}{0}
   \barbil{8}{3}{0}
   \tige{9}{5}{1}
   \binoire{9}{10}{gray}
   \tige{10}{0}{0}
   \tige{11}{0}{0}
   \cadre{11}
\end{tikzpicture}
\end{minipage}%
%
\begin{minipage}{0.5\textwidth}
\begin{tikzpicture}
   \tige{1}{0}{0}
   \tige{2}{0}{0}       %<- here a dot: unit million
   \tige{3}{0}{1}       %<- here no dot: hundreds of thousands
   \tige{4}{0}{0}       %<- here no dot: tens of thousands
   \tige{5}{0}{0}       %<- here a dot: units thousands
   \tige{6}{0}{1}
   \tige{7}{1}{0}
   \binoire{7}{5}{gray}
   \tige{8}{5}{0}      
   \binoire{8}{10}{gray}
   \tige{9}{7}{1}         %<- here a dot: units
   \binoire{9}{10}{gray}
   \binoire{9}{5}{gray}
   \binoire{9}{4}{gray}
   \tige{10}{0}{0}
   \tige{11}{0}{0}
   \cadre{11}
\end{tikzpicture}
\end{minipage}


\paragraph{87+65= 152.} Marque 87 no ábaco.
Em seguida, some 100-50+10= 60:\\

\vspace{0.2cm}
\begin{minipage}{0.4\textwidth}
\begin{tikzpicture}
   \tige{1}{0}{0}
   \tige{2}{0}{0}       %<- here a dot: unit million
   \tige{3}{0}{1}       %<- here no dot: hundreds of thousands
   \tige{4}{0}{0}       %<- here no dot: tens of thousands
   \tige{5}{0}{0}       %<- here a dot: units thousands
   \tige{6}{0}{1}
   \tige{7}{0}{0}
   \tige{8}{8}{0}       %<- here a dot: units
   \binoire{8}{10}{gray}
   \binoire{8}{5}{gray}
   \binoire{8}{4}{gray}
   \binoire{8}{3}{gray}
   \tige{9}{7}{1}
   \binoire{9}{10}{gray}
   \binoire{9}{5}{gray}
   \binoire{9}{4}{gray}
   \tige{10}{0}{0}
   \tige{11}{0}{0}
   \cadre{11}
\end{tikzpicture}
\end{minipage}%
%
\begin{minipage}{0.5\textwidth}
\begin{tikzpicture}
   \tige{1}{0}{0}
   \tige{2}{0}{0}       %<- here a dot: unit million
   \tige{3}{0}{1}       %<- here no dot: hundreds of thousands
   \tige{4}{0}{0}       %<- here no dot: tens of thousands
   \tige{5}{0}{0}       %<- here a dot: units thousands
   \tige{6}{0}{1}
   \tige{7}{1}{0}
   \binoire{7}{5}{gray}
   \tige{8}{4}{0}       %<- here a dot: units
   \binoire{8}{2}{gray}
   \binoire{8}{3}{gray}
   \binoire{8}{4}{gray}
   \binoire{8}{5}{gray}
   \barbil{8}{9}{0}
   \tige{9}{7}{1}
   \binoire{9}{10}{gray}
   \binoire{9}{5}{gray}
   \binoire{9}{4}{gray}
   \tige{10}{0}{0}
   \tige{11}{0}{0}
   \cadre{11}
\end{tikzpicture}
\end{minipage}

\verb||\\
Agora temos de somar 5 nas unidades. Acontece
que só temos duas pedras disponíveis
na coluna das unidades. O que fazer?
Somamos 10 e subtraímos 5. \\

\begin{minipage}{0.4\textwidth}
\begin{tikzpicture}
   \tige{1}{0}{0}
   \tige{2}{0}{0}       %<- here a dot: unit million
   \tige{3}{0}{1}       %<- here no dot: hundreds of thousands
   \tige{4}{0}{0}       %<- here no dot: tens of thousands
   \tige{5}{0}{0}       %<- here a dot: units thousands
   \tige{6}{0}{1}
   \tige{7}{1}{0}
   \binoire{7}{5}{gray}
   \tige{8}{5}{0}       %<- here a dot: units
   \binoire{8}{10}{gray}
   \barbil{8}{4}{0}
   \barbil{8}{3}{0}
   \barbil{8}{2}{0}
   \barbil{8}{1}{0}
   \tige{9}{2}{1}
   \binoire{9}{5}{gray}
   \binoire{9}{4}{gray}
   \barbil{9}{9}{1}
   \tige{10}{0}{0}
   \tige{11}{0}{0}
   \cadre{11}
\end{tikzpicture}
\end{minipage}%
%
\begin{minipage}{0.5\textwidth}
Para somar 10, foi preciso somar
a pedra de 50 (em cinza) e subtrair
40 (4 pedras cortadas nas dezenas).
Para subtrair 5, bastou eliminar a pedra de 5 unidades,
que aparece cortada.
\end{minipage}


\section{Soma de três dígitos}
Vamos verificar que 568+947= 1515. Inicialmente,
vamos marcar 568 no ábaco.\\

\begin{tikzpicture}
   \tige{1}{0}{0}
   \tige{2}{0}{0}       %<- here a dot: unit million
   \tige{3}{0}{1}       %<- here no dot: hundreds of thousands
   \tige{4}{0}{0}       %<- here no dot: tens of thousands
   \tige{5}{0}{0}       %<- here a dot: units thousands
   \tige{6}{0}{1}
   \tige{7}{5}{0}       %% milhar
   \tige{8}{6}{0}       %<- here a dot: units
   \tige{9}{8}{1}
   \binoire{7}{10}{gray}
   \binoire{8}{10}{gray}
   \binoire{8}{5}{gray}
   \binoire{9}{10}{gray}
   \binoire{9}{5}{gray}
   \binoire{9}{4}{gray}
   \binoire{9}{3}{gray}   
   \tige{10}{0}{0}
   \tige{11}{0}{0}
   \cadre{11}
\end{tikzpicture}

Agora, vamos somar 900. Como não há pedras suficientes
na coluna das centenas, precisamos de somar 1000
e subtrair 100. Para subtrair 100, temos de
subtrair 500 e somar 400. No diagrama abaixo,
a pedra de 500 que foi subtraída das dezenas
aparece cortada. As pedras que representam
dígitos aparecem em cinza. Note que, até aqui,
somamos 568 com 900, e o resultado 1468
aparece claramente no ábaco.

\begin{tikzpicture}
   \tige{1}{0}{0}
   \tige{2}{0}{0}       %<- here a dot: unit million
   \tige{3}{0}{1}       %<- here no dot: hundreds of thousands
   \tige{4}{0}{0}       %<- here no dot: tens of thousands
   \tige{5}{0}{0}       %<- here a dot: units thousands
   \tige{6}{1}{1}       %% milhar
   \tige{7}{4}{0}       %% centena   
   \tige{8}{6}{0}       %<- here a dot: units
   \tige{9}{8}{1}
   \tige{10}{0}{0}
   \tige{11}{0}{0}
   \binoire{6}{5}{gray}
   \barbil{7}{9}{0}
   \binoire{7}{2}{gray}
   \binoire{7}{3}{gray}
   \binoire{7}{4}{gray}
   \binoire{7}{5}{gray}
   \binoire{8}{10}{gray}
   \binoire{8}{5}{gray}
   \binoire{9}{10}{gray}
   \binoire{9}{5}{gray}
   \binoire{9}{4}{gray}
   \binoire{9}{3}{gray}
   \cadre{11}
\end{tikzpicture}

Agora vamos somar mais 40 a 1468, onde 40 são as
dezenas de 947. Como não temos pedras suficientes
na casa das dezenas, vamos somar 100 e subtrair
60. Para somar 100, precisamos de adicionar a
pedra de 500 e subtrair 400. A subtração de
60 é direta, pois há uma pedra de 10 e uma de
50 em 1468. As pedras que
formam os 400 eliminados estão cortadas.
O resultado até agora está em 1508.\\

\begin{tikzpicture}
   \tige{1}{0}{0}
   \tige{2}{0}{0}       %<- here a dot: unit million
   \tige{3}{0}{1}       %<- here no dot: hundreds of thousands
   \tige{4}{0}{0}       %<- here no dot: tens of thousands
   \tige{5}{0}{0}       %<- here a dot: units thousands
   \tige{6}{1}{1}       %% milhar
   \tige{7}{5}{0}       %% centena
   \tige{8}{0}{0}       %<- here a dot: units
   \tige{9}{8}{1}
   \tige{10}{0}{0}
   \tige{11}{0}{0}
   \binoire{6}{5}{gray}
   \binoire{7}{10}{gray}
   \barbil{7}{4}{0}
   \barbil{7}{3}{0}
   \barbil{7}{2}{0}
   \barbil{7}{1}{0}
   \binoire{9}{10}{gray}
   \binoire{9}{5}{gray}
   \binoire{9}{4}{gray}
   \binoire{9}{3}{gray}
   \cadre{11}
\end{tikzpicture}

Para terminar o cálculo, falta apenas somar 7
a 1508. Como sói acontecer, não temos pedras
suficientes nas unidades para adicionar 7.
Então, somamos 10 e subtraímos 3.\\

\begin{tikzpicture}
   \tige{1}{0}{0}
   \tige{2}{0}{0}       %<- here a dot: unit million
   \tige{3}{0}{1}       %<- here no dot: hundreds of thousands
   \tige{4}{0}{0}       %<- here no dot: tens of thousands
   \tige{5}{0}{0}       %<- here a dot: units thousands
   \tige{6}{1}{1}       %% milhar
   \tige{7}{5}{0}       %% centena
   \tige{8}{1}{0}       %<- here a dot: units
   \tige{9}{5}{1}
   \tige{10}{0}{0}
   \tige{11}{0}{0}
   \barbil{9}{2}{1}
   \barbil{9}{3}{1}
   \barbil{9}{4}{1}
   \binoire{8}{5}{gray}
   \binoire{6}{5}{gray}
   \binoire{7}{10}{gray}
   \binoire{8}{5}{gray}
   \binoire{9}{10}{gray}
   \cadre{11}
\end{tikzpicture}

O resultado final da soma 568+947 é 1515
e aparece em cinza na figura abaixo.\\

\ladj{0.5}
\begin{tikzpicture}
   \tige{1}{0}{0}
   \tige{2}{0}{0}       %<- here a dot: unit million
   \tige{3}{0}{1}       %<- here no dot: hundreds of thousands
   \tige{4}{0}{0}       %<- here no dot: tens of thousands
   \tige{5}{0}{0}       %<- here a dot: units thousands
   \tige{6}{1}{1}       %% milhar
   \tige{7}{5}{0}       %% centena
   \tige{8}{1}{0}       %<- here a dot: units
   \tige{9}{5}{1}
   \tige{10}{0}{0}
   \tige{11}{0}{0}
   \binoire{6}{5}{gray}
   \binoire{7}{10}{gray}
   \binoire{8}{5}{gray}
   \binoire{9}{10}{gray}
   \cadre{11}
\end{tikzpicture}


\section{Adições encadeadas}

Vamos ver como fazer a seguinte sequência
de somas encadeadas: 34+87+38+95.\\

\ladj{0.25}
\begin{minipage}{0.4\textwidth}
\begin{tikzpicture}
   \tige{1}{0}{0}
   \tige{2}{0}{0}       %<- here a dot: unit million
   \tige{3}{0}{1}       %<- here no dot: hundreds of thousands
   \tige{4}{0}{0}       %<- here no dot: tens of thousands
   \tige{5}{0}{0}       %<- here a dot: units thousands
   \tige{6}{0}{1}       %% milhar
   \tige{7}{0}{0}       %% centena
   \tige{8}{3}{0}       %% dezena
   \tige{9}{4}{1}       %% unidade
   \tige{10}{0}{0}
   \tige{11}{0}{0}
   \cadre{11}
\end{tikzpicture}
\end{minipage}%
%
\begin{minipage}{0.5\textwidth}
  Começamos marcando 34 no ábaco.
  Adicionamos 30 na casa das
  dezenas e 4 nas unidades,
  conforme indicado ao lado.
\end{minipage}


\vspace{0.5cm}
\paragraph{34+87}= 121\\

\begin{minipage}{0.4\textwidth}
\begin{tikzpicture}
   \tige{1}{0}{0}
   \tige{2}{0}{0}       %<- here a dot: unit million
   \tige{3}{0}{1}       %<- here no dot: hundreds of thousands
   \tige{4}{0}{0}       %<- here no dot: tens of thousands
   \tige{5}{0}{0}       %<- here a dot: units thousands
   \tige{6}{0}{1}       %% milhar
   \tige{7}{1}{0}       %% centena
   \tige{8}{1}{0}       %% dezena
   \tige{9}{4}{1}       %% unidade
   \tige{10}{0}{0}
   \tige{11}{0}{0}
   \cadre{11}
\end{tikzpicture}
\end{minipage}%
%
\begin{minipage}{0.5\textwidth}
  Para efetuar 87+34, começamos somando 80+34.
  Como não temos pedras suficientes nas
  dezenas, somamos 100 e subtraímos 20.
\end{minipage}

\vspace{0.5cm}
\begin{minipage}{0.4\textwidth}
\begin{tikzpicture}
   \tige{1}{0}{0}
   \tige{2}{0}{0}       %<- here a dot: unit million
   \tige{3}{0}{1}       %<- here no dot: hundreds of thousands
   \tige{4}{0}{0}       %<- here no dot: tens of thousands
   \tige{5}{0}{0}       %<- here a dot: units thousands
   \tige{6}{0}{1}       %% milhar
   \tige{7}{1}{0}       %% centena
   \tige{8}{2}{0}       %% dezena
   \tige{9}{1}{1}       %% unidade
   \tige{10}{0}{0}
   \tige{11}{0}{0}
   \cadre{11}
\end{tikzpicture}
\end{minipage}%
%
\begin{minipage}{0.5\textwidth}
  Agora somamos 7 na casa das unidades.
  Na situação do ábaco, isso corresponde
  a somar 10 e subtrair 3, como mostrado
  ao lado.
\end{minipage}


\vspace{0.5cm}
\paragraph{121+38}=159\\

\begin{minipage}{0.4\textwidth}
\begin{tikzpicture}
   \tige{1}{0}{0}
   \tige{2}{0}{0}       %<- here a dot: unit million
   \tige{3}{0}{1}       %<- here no dot: hundreds of thousands
   \tige{4}{0}{0}       %<- here no dot: tens of thousands
   \tige{5}{0}{0}       %<- here a dot: units thousands
   \tige{6}{0}{1}       %% milhar
   \tige{7}{1}{0}       %% centena
   \tige{8}{5}{0}       %% dezena
   \tige{9}{1}{1}       %% unidade
   \tige{10}{0}{0}
   \tige{11}{0}{0}
   \cadre{11}
\end{tikzpicture}
\end{minipage}%
%
\begin{minipage}{0.5\textwidth}
  Para somar 38 a 121, começamos somando 30,
  ou seja, adicionamos 50 e subtraímos 20
  da casa das dezenas. O resultado é 151.
\end{minipage}

\vspace{0.4cm}
\begin{minipage}{0.4\textwidth}
\begin{tikzpicture}
   \tige{1}{0}{0}
   \tige{2}{0}{0}       %<- here a dot: unit million
   \tige{3}{0}{1}       %<- here no dot: hundreds of thousands
   \tige{4}{0}{0}       %<- here no dot: tens of thousands
   \tige{5}{0}{0}       %<- here a dot: units thousands
   \tige{6}{0}{1}       %% milhar
   \tige{7}{1}{0}       %% centena
   \tige{8}{5}{0}       %% dezena
   \tige{9}{9}{1}       %% unidade
   \tige{10}{0}{0}
   \tige{11}{0}{0}
   \cadre{11}
\end{tikzpicture}
\end{minipage}%
%
\begin{minipage}{0.5\textwidth}
  Agora somamos 8 na casa das unidades,
  isto é, 5+3. O resultado é 159.
\end{minipage}


\vspace{0.8cm}
\paragraph{159+95}= 254\\

\begin{minipage}{0.4\textwidth}
\begin{tikzpicture}
   \tige{1}{0}{0}
   \tige{2}{0}{0}       %<- here a dot: unit million
   \tige{3}{0}{1}       %<- here no dot: hundreds of thousands
   \tige{4}{0}{0}       %<- here no dot: tens of thousands
   \tige{5}{0}{0}       %<- here a dot: units thousands
   \tige{6}{0}{1}       %% milhar
   \tige{7}{2}{0}       %% centena
   \tige{8}{4}{0}       %% dezena
   \tige{9}{9}{1}       %% unidade
   \tige{10}{0}{0}
   \tige{11}{0}{0}
   \cadre{11}
\end{tikzpicture}
\end{minipage}%
%
\begin{minipage}{0.5\textwidth}
  Para somar 95 a 159, vamos
  inicialmente somar 90 a 159.
  Devido à alocação das pedras na
  hari de dezenas, essa operação
  corresponde a somar 100,
  subtrair 50 e adicionar 4
  pedras de 10. O resultado mostrado
  na figura é 249.
\end{minipage}


\begin{minipage}{0.4\textwidth}
\begin{tikzpicture}
   \tige{1}{0}{0}
   \tige{2}{0}{0}       %<- here a dot: unit million
   \tige{3}{0}{1}       %<- here no dot: hundreds of thousands
   \tige{4}{0}{0}       %<- here no dot: tens of thousands
   \tige{5}{0}{0}       %<- here a dot: units thousands
   \tige{6}{0}{1}       %% milhar
   \tige{7}{2}{0}       %% centena
   \tige{8}{5}{0}       %% dezena
   \tige{9}{4}{1}       %% unidade
   \tige{10}{0}{0}
   \tige{11}{0}{0}
   \cadre{11}
\end{tikzpicture}
\end{minipage}%
%
\begin{minipage}{0.5\textwidth}
\verb||\\
  Para somar 5 unidades ao número
  249 que está no soroban, temos
  que acrescentar 10 à coluna das
  dezenas. A única maneira de
  fazer isso é somar 50 e subtrair
  40. O próximo passo é subtrair
  5 da coluna das unidades.
\end{minipage}

\section{Soma com transbordamento}

Nessa seção, vamos estudar as ocorrências
de soma de dígitos. Em suma, queremos saber
o que pode acontecer quando tentamos somar
dois dígitos. As possibilidades são:

\paragraph{Há pedras para o segundo dígito.}
Depois de marcar o primeiro dígito, há pedras
livres em quantidade suficiente para representar
o segundo dígito. No exemplo abaixo, fizemos a
soma de $3+4$. O primeiro
dígito é 3, de modo que sobram a pedra de 5
e uma das pedras de 1 para adicionar 4. Então,
adicionamos a pedra de 5 e subtraímos 1.
A resposta é 7.\\

\vspace{0.5cm}
\begin{minipage}{0.4\textwidth}
\begin{tikzpicture}
   \tige{1}{0}{0}
   \tige{2}{0}{0}       %<- here a dot: unit million
   \tige{3}{0}{1}       %<- here no dot: hundreds of thousands
   \tige{4}{0}{0}       %<- here no dot: tens of thousands
   \tige{5}{0}{0}       %<- here a dot: units thousands
   \tige{6}{0}{1}       %% milhar
   \tige{7}{0}{0}       %% centena
   \tige{8}{0}{0}       %% dezena
   \tige{9}{3}{1}       %% unidade
   \binoire{9}{5}{gray}
   \binoire{9}{4}{gray}
   \binoire{9}{3}{gray}
   \tige{10}{0}{0}
   \tige{11}{0}{0}
   \cadre{11}
\end{tikzpicture}
\end{minipage}%
%
\begin{minipage}{0.5\textwidth}
\begin{tikzpicture}
   \tige{1}{0}{0}
   \tige{2}{0}{0}       %<- here a dot: unit million
   \tige{3}{0}{1}       %<- here no dot: hundreds of thousands
   \tige{4}{0}{0}       %<- here no dot: tens of thousands
   \tige{5}{0}{0}       %<- here a dot: units thousands
   \tige{6}{0}{1}       %% milhar
   \tige{7}{0}{0}       %% centena
   \tige{8}{0}{0}       %% dezena
   \tige{9}{7}{1}       %% unidade
   \binoire{9}{10}{gray}
   \binoire{9}{5}{gray}
   \binoire{9}{4}{gray}
   \barbil{9}{2}{0}
   \tige{10}{0}{0}
   \tige{11}{0}{0}
   \cadre{11}
\end{tikzpicture}
\end{minipage}

\vspace{0.5cm}
\paragraph{Não há pedras na coluna para o segundo dígito.}
Em pouco mais da metade dos casos, o número de pedras
não é suficiente para adicionar o segundo dígito.
Suponhamos que você queira somar $D_1+D_2$ na
coluna $n$ e descobre que não tem pedras disponíveis
para $D_2$. Nesse caso, some $10^{n+1}$ à coluna $n+1$
e subtrai $(10-D_2)\times 10^n$ da coluna $n$.

Tomemos um exemplo concreto. Você vai somar $D_1=\;6$
com $D_2=\;8$. Depois de marcar
$D_1=\;6$ na coluna 0, você verifica que
não há pedras bastante
para somar $D_2=\;8$. Então, você soma $1\times 10^{1}$ à coluna 1
e subtrai $(10-8)\times 10^0=\;2$ da coluna 0. Isso é equivalente
a somar $8$ à coluna 0. O resultado é 14.

\vspace{0.5cm}
\begin{minipage}{0.4\textwidth}
\begin{tikzpicture}
   \tige{1}{0}{0}
   \tige{2}{0}{0}       %<- here a dot: unit million
   \tige{3}{0}{1}       %<- here no dot: hundreds of thousands
   \tige{4}{0}{0}       %<- here no dot: tens of thousands
   \tige{5}{0}{0}       %<- here a dot: units thousands
   \tige{6}{0}{1}       %% milhar
   \tige{7}{0}{0}       %% centena
   \tige{8}{0}{0}       %% dezena
   \tige{9}{6}{1}       %% unidade
   \tige{10}{0}{0}
   \tige{11}{0}{0}
   \binoire{9}{5}{gray}
   \binoire{9}{10}{gray}
   \cadre{11}
\end{tikzpicture}
\end{minipage}%
%
\begin{minipage}{0.5\textwidth}
\begin{tikzpicture}
   \tige{1}{0}{0}
   \tige{2}{0}{0}       %<- here a dot: unit million
   \tige{3}{0}{1}       %<- here no dot: hundreds of thousands
   \tige{4}{0}{0}       %<- here no dot: tens of thousands
   \tige{5}{0}{0}       %<- here a dot: units thousands
   \tige{6}{0}{1}       %% milhar
   \tige{7}{0}{0}       %% centena
   \tige{8}{1}{0}       %% dezena
   \binoire{8}{5}{gray}
   \tige{9}{4}{1}       %% unidade
   \tige{10}{0}{0}
   \binoire{9}{5}{gray}
   \tige{11}{4}{0}
   \binoire{9}{5}{gray}
   \binoire{9}{4}{gray}
   \binoire{9}{3}{gray}
   \binoire{9}{2}{gray}
   \barbil{9}{9}{0}
   \cadre{11}
\end{tikzpicture}
\end{minipage}

\vspace{0.5cm}
Na fase de subtrair $2$ da 
coluna 0, houve um pequeno contratempo:
não havia $2$ pedras de
valor unitário para efetuar a subtração.
Por isso, subtraímos a pedra de 5
e somamos 3 pedras de 1. 

Você deve ter notado a estranha maneira
que estamos usando para numerar as colunas.
A coluna 0 é a terceira da esquerda para
a direita. A coluna 0 é a coluna das
unidades. A primeira coluna da esquerda
é a coluna $-2$ e equivale aos centésimos.
A segunda coluna é a $-1$ e representa
os décimos. A pedra de 5 representa $0.5$
na coluna $-1$.

Para realmente acompanhar essa explicação,
você precisa entender que 1 na coluna
$n$ é igual a $10^n$. Então, quando você
soma $1$ na coluna $1$, está realmente
somando $10$. Para apreciar esse fato em
sua totalidade, vamos supor que você
queira somar $600+800$. Nada mudou no
problema, exceto que você vai trabalhar
na coluna 2.

\vspace{0.5cm}
\begin{minipage}{0.4\textwidth}
\begin{tikzpicture}
   \tige{1}{0}{0}
   \tige{2}{0}{0}       %<- here a dot: unit million
   \tige{3}{0}{1}       %<- here no dot: hundreds of thousands
   \tige{4}{0}{0}       %<- here no dot: tens of thousands
   \tige{5}{0}{0}       %<- here a dot: units thousands
   \tige{6}{0}{1}       %% milhar
   \tige{7}{6}{0}       %% centena
   \binoire{7}{5}{gray}
   \binoire{7}{10}{gray}
   \tige{8}{0}{0}       %% dezena
   \tige{9}{0}{1}       %% unidade
   \tige{10}{0}{0}
   \tige{11}{0}{0}
   \cadre{11}
\end{tikzpicture}
\end{minipage}%
%
\begin{minipage}{0.5\textwidth}
\begin{tikzpicture}
   \tige{1}{0}{0}
   \tige{2}{0}{0}       %<- here a dot: unit million
   \tige{3}{0}{1}       %<- here no dot: hundreds of thousands
   \tige{4}{0}{0}       %<- here no dot: tens of thousands
   \tige{5}{0}{0}       %<- here a dot: units thousands
   \tige{6}{1}{1}       %% milhar
   \tige{7}{4}{0}       %% centena
   \tige{8}{0}{0}       %% dezena
   \tige{9}{0}{1}       %% unidade
   \tige{10}{0}{0}
   \binoire{6}{5}{gray}
   \tige{11}{0}{0}
   \binoire{7}{5}{gray}
   \binoire{7}{4}{gray}
   \binoire{7}{3}{gray}
   \binoire{7}{2}{gray}
   \barbil{7}{9}{0}
   \cadre{11}
\end{tikzpicture}
\end{minipage}

\section{Coluna de transbordamento cheia}
Conforme já vimos, há situações em
que é impossível somar um dígito $D_1$
ao conteúdo da coluna $n$. Quando isso
acontece, temos de incrementar a coluna
$n+1$. Entretanto, a coluna $n+1$ pode
já ter 9 pontos. Nesse caso, zeramos
a coluna $n+1$ e incrementamos
a coluna $n+2$ de $1$. Vejamos um
exemplo concreto. Vamos somar $86+17$.
Essa soma pode ser decomposta em
$86+10+7$. Quando somamos $10$ a $86$,
a coluna das dezenas fica cheia com
$9$ dezenas.

\vspace{0.2cm}
\begin{minipage}{0.4\textwidth}
\begin{tikzpicture}
   \tige{1}{0}{0}
   \tige{2}{0}{0}       %<- here a dot: unit million
   \tige{3}{0}{1}       %<- here no dot: hundreds of thousands
   \tige{4}{0}{0}       %<- here no dot: tens of thousands
   \tige{5}{0}{0}       %<- here a dot: units thousands
   \tige{6}{0}{1}       %% milhar
   \tige{7}{0}{0}       %% centena
   \tige{8}{8}{0}       %% dezena
   \tige{9}{6}{1}       %% unidade
   \tige{10}{0}{0}
   \tige{11}{0}{0}
   \cadre{11}
\end{tikzpicture}
\end{minipage}%
%
\begin{minipage}{0.5\textwidth}
\begin{tikzpicture}
   \tige{1}{0}{0}
   \tige{2}{0}{0}       %<- here a dot: unit million
   \tige{3}{0}{1}       %<- here no dot: hundreds of thousands
   \tige{4}{0}{0}       %<- here no dot: tens of thousands
   \tige{5}{0}{0}       %<- here a dot: units thousands
   \tige{6}{0}{1}       %% milhar
   \tige{7}{0}{0}       %% centena
   \tige{8}{9}{0}       %% dezena
   \tige{9}{6}{1}       %% unidade
   \tige{10}{0}{0}
   \tige{11}{0}{0}
   \cadre{11}
\end{tikzpicture}
\end{minipage}

\vspace{0.2cm}
Quando tentamos somar $7$, não temos
pedras suficientes na coluna das unidades.
E quando transbordamos 10 para a coluna
das dezenas, descobrimos que essa coluna
está cheia. Então zeramos a coluna das
dezenas e acrescentamos 100 à coluna
das centenas. Em seguida, subtraímos 3
da coluna das unidades. A resposta é 103.

\vspace{0.2cm}
\begin{minipage}{0.4\textwidth}
\begin{tikzpicture}
   \tige{1}{0}{0}
   \tige{2}{0}{0}       %<- here a dot: unit million
   \tige{3}{0}{1}       %<- here no dot: hundreds of thousands
   \tige{4}{0}{0}       %<- here no dot: tens of thousands
   \tige{5}{0}{0}       %<- here a dot: units thousands
   \tige{6}{0}{1}       %% milhar
   \tige{7}{0}{0}       %% centena
   \tige{8}{9}{0}       %% dezena
   \tige{9}{6}{1}       %% unidade
   \tige{10}{0}{0}
   \tige{11}{0}{0}
   \cadre{11}
\end{tikzpicture}
\end{minipage}%
%
\begin{minipage}{0.5\textwidth}
\begin{tikzpicture}
   \tige{1}{0}{0}
   \tige{2}{0}{0}       %<- here a dot: unit million
   \tige{3}{0}{1}       %<- here no dot: hundreds of thousands
   \tige{4}{0}{0}       %<- here no dot: tens of thousands
   \tige{5}{0}{0}       %<- here a dot: units thousands
   \tige{6}{0}{1}       %% milhar
   \tige{7}{1}{0}       %% centena
   \tige{8}{0}{0}       %% dezena
   \tige{9}{3}{1}       %% unidade
   \tige{10}{0}{0}
   \tige{11}{0}{0}
   \cadre{11}
\end{tikzpicture}
\end{minipage}

\vspace{0.2cm}
É possível encontrar uma sequência de duas ou mais
colunas com 9 pontos cada coluna. Se houver transbordamento para
a primeira dessas colunas, devemos zerar todas,
até encontrar uma que possa receber o transbordamento.

\vspace{0.2cm}
\begin{minipage}{0.4\textwidth}
\begin{tikzpicture}
   \tige{1}{0}{0}
   \tige{2}{0}{0}       %<- here a dot: unit million
   \tige{3}{0}{1}       %<- here no dot: hundreds of thousands
   \tige{4}{0}{0}       %<- here no dot: tens of thousands
   \tige{5}{0}{0}       %<- here a dot: units thousands
   \tige{6}{9}{1}       %% milhar
   \tige{7}{9}{0}       %% centena
   \tige{8}{9}{0}       %% dezena
   \tige{9}{7}{1}       %% unidade
   \tige{10}{0}{0}
   \tige{11}{0}{0}
   \cadre{11}
\end{tikzpicture}
\end{minipage}%
%
\begin{minipage}{0.5\textwidth}
\begin{tikzpicture}
   \tige{1}{0}{0}
   \tige{2}{0}{0}       %<- here a dot: unit million
   \tige{3}{0}{1}       %<- here no dot: hundreds of thousands
   \tige{4}{0}{0}       %<- here no dot: tens of thousands
   \tige{5}{1}{0}       %<- here a dot: units thousands
   \tige{6}{0}{1}       %% milhar
   \tige{7}{0}{0}       %% centena
   \tige{8}{0}{0}       %% dezena
   \tige{9}{5}{1}       %% unidade
   \tige{10}{0}{0}
   \tige{11}{0}{0}
   \cadre{11}
\end{tikzpicture}
\end{minipage}


\section{Exercícios de soma}

\paragraph{505+4016}= 4521.\\

\begin{minipage}{0.4\textwidth}
\begin{tikzpicture}
   \tige{1}{0}{0}
   \tige{2}{0}{0}       %<- here a dot: unit million
   \tige{3}{0}{1}       %<- here no dot: hundreds of thousands
   \tige{4}{0}{0}       %<- here no dot: tens of thousands
   \tige{5}{0}{0}       %<- here a dot: units thousands
   \tige{6}{0}{1}       %% milhar
   \tige{7}{5}{0}       %% centena
   \tige{8}{0}{0}       %% dezena
   \tige{9}{5}{1}       %% unidade
   \tige{10}{0}{0}
   \tige{11}{0}{0}
   \cadre{11}
\end{tikzpicture}
\end{minipage}%
%
\begin{minipage}{0.5\textwidth}
\begin{tikzpicture}
   \tige{1}{0}{0}
   \tige{2}{0}{0}       %<- here a dot: unit million
   \tige{3}{0}{1}       %<- here no dot: hundreds of thousands
   \tige{4}{0}{0}       %<- here no dot: tens of thousands
   \tige{5}{0}{0}       %<- here a dot: units thousands
   \tige{6}{4}{1}       %% milhar
   \tige{7}{5}{0}       %% centena
   \tige{8}{2}{0}       %% dezena
   \tige{9}{1}{1}       %% unidade
   \tige{10}{0}{0}
   \tige{11}{0}{0}
   \cadre{11}
\end{tikzpicture}
\end{minipage}

\vspace{0.2cm}
\paragraph{2751+5221}= 7972\\

\vspace{0.4cm}
\begin{minipage}{0.4\textwidth}
\begin{tikzpicture}
   \tige{1}{0}{0}
   \tige{2}{0}{0}       %<- here a dot: unit million
   \tige{3}{0}{1}       %<- here no dot: hundreds of thousands
   \tige{4}{0}{0}       %<- here no dot: tens of thousands
   \tige{5}{0}{0}       %<- here a dot: units thousands
   \tige{6}{2}{1}       %% milhar
   \tige{7}{7}{0}       %% centena
   \tige{8}{5}{0}       %% dezena
   \tige{9}{1}{1}       %% unidade
   \tige{10}{0}{0}
   \tige{11}{0}{0}
   \cadre{11}
\end{tikzpicture}
\end{minipage}%
%
\begin{minipage}{0.5\textwidth}
\begin{tikzpicture}
   \tige{1}{0}{0}
   \tige{2}{0}{0}       %<- here a dot: unit million
   \tige{3}{0}{1}       %<- here no dot: hundreds of thousands
   \tige{4}{0}{0}       %<- here no dot: tens of thousands
   \tige{5}{0}{0}       %<- here a dot: units thousands
   \tige{6}{7}{1}       %% milhar
   \tige{7}{9}{0}       %% centena
   \tige{8}{7}{0}       %% dezena
   \tige{9}{2}{1}       %% unidade
   \tige{10}{0}{0}
   \tige{11}{0}{0}
   \cadre{11}
\end{tikzpicture}
\end{minipage}


\vspace{1cm}
\paragraph{5997+8029}= 14026\\

\begin{minipage}{0.4\textwidth}
\begin{tikzpicture}
   \tige{1}{0}{0}
   \tige{2}{0}{0}       %<- here a dot: unit million
   \tige{3}{0}{1}       %<- here no dot: hundreds of thousands
   \tige{4}{0}{0}       %<- here no dot: tens of thousands
   \tige{5}{0}{0}       %<- here a dot: units thousands
   \tige{6}{5}{1}       %% milhar
   \tige{7}{9}{0}       %% centena
   \tige{8}{9}{0}       %% dezena
   \tige{9}{7}{1}       %% unidade
   \tige{10}{0}{0}
   \tige{11}{0}{0}
   \cadre{11}
\end{tikzpicture}
\end{minipage}%
%
\begin{minipage}{0.5\textwidth}
\begin{tikzpicture}
   \tige{1}{0}{0}
   \tige{2}{0}{0}       %<- here a dot: unit million
   \tige{3}{0}{1}       %<- here no dot: hundreds of thousands
   \tige{4}{0}{0}       %<- here no dot: tens of thousands
   \tige{5}{1}{0}       %<- here a dot: units thousands
   \tige{6}{4}{1}       %% milhar
   \tige{7}{0}{0}       %% centena
   \tige{8}{2}{0}       %% dezena
   \tige{9}{6}{1}       %% unidade
   \tige{10}{0}{0}
   \tige{11}{0}{0}
   \cadre{11}
\end{tikzpicture}
\end{minipage}

\vspace{0.5cm}
\paragraph{3932+5297+7320}= 16549\\

\vspace{0.4cm}
\begin{minipage}{0.3\textwidth}
\begin{tikzpicture}
   \tige{1}{0}{0}
   \tige{2}{0}{0}       %<- here a dot: unit million
   \tige{3}{3}{1}       %<- here no dot: hundreds of thousands
   \tige{4}{9}{0}       %<- here no dot: tens of thousands
   \tige{5}{3}{0}       %<- here a dot: units thousands
   \tige{6}{2}{1}       %% milhar
   \tige{7}{0}{0}       %% centena
   \tige{8}{0}{0}       %% dezena
   \cadre{8}
\end{tikzpicture}
\end{minipage}%
%
\begin{minipage}{0.3\textwidth}
\begin{tikzpicture}
   \tige{1}{0}{0}
   \tige{2}{0}{0}       %<- here a dot: unit million
   \tige{3}{9}{1}       %<- here no dot: hundreds of thousands
   \tige{4}{2}{0}       %<- here no dot: tens of thousands
   \tige{5}{2}{0}       %<- here a dot: units thousands
   \tige{6}{9}{1}       %% milhar
   \tige{7}{0}{0}       %% centena
   \tige{8}{0}{0}       %% dezena
   \cadre{8}
\end{tikzpicture}
\end{minipage}%
%
\begin{minipage}{0.3\textwidth}
\begin{tikzpicture}
   \tige{1}{0}{0}
   \tige{2}{1}{0}       %<- here a dot: unit million
   \tige{3}{6}{1}       %<- here no dot: hundreds of thousands
   \tige{4}{5}{0}       %<- here no dot: tens of thousands
   \tige{5}{4}{0}       %<- here a dot: units thousands
   \tige{6}{9}{1}       %% milhar
   \tige{7}{0}{0}       %% centena
   \tige{8}{0}{0}       %% dezena
   \cadre{8}
\end{tikzpicture}
\end{minipage}

\vspace{0.4cm}
\paragraph{3777+9979+9886}= \\

\vspace{0.2cm}
\begin{minipage}{0.3\textwidth}
\begin{tikzpicture}
   \tige{1}{0}{0}
   \tige{2}{0}{0}       %<- here a dot: unit million
   \tige{3}{3}{1}       %<- here no dot: hundreds of thousands
   \tige{4}{7}{0}       %<- here no dot: tens of thousands
   \tige{5}{7}{0}       %<- here a dot: units thousands
   \tige{6}{7}{1}       %% milhar
   \tige{7}{0}{0}       %% centena
   \tige{8}{0}{0}       %% dezena
   \cadre{8}
\end{tikzpicture}
\end{minipage}%
%
\begin{minipage}{0.3\textwidth}
\begin{tikzpicture}
   \tige{1}{0}{0}
   \tige{2}{1}{0}       %<- here a dot: unit million
   \tige{3}{3}{1}       %<- here no dot: hundreds of thousands
   \tige{4}{7}{0}       %<- here no dot: tens of thousands
   \tige{5}{5}{0}       %<- here a dot: units thousands
   \tige{6}{6}{1}       %% milhar
   \tige{7}{0}{0}       %% centena
   \tige{8}{0}{0}       %% dezena
   \cadre{8}
\end{tikzpicture}
\end{minipage}%
%
\begin{minipage}{0.3\textwidth}
\begin{tikzpicture}
   \tige{1}{0}{0}
   \tige{2}{2}{0}       %<- here a dot: unit million
   \tige{3}{3}{1}       %<- here no dot: hundreds of thousands
   \tige{4}{6}{0}       %<- here no dot: tens of thousands
   \tige{5}{4}{0}       %<- here a dot: units thousands
   \tige{6}{2}{1}       %% milhar
   \tige{7}{0}{0}       %% centena
   \tige{8}{0}{0}       %% dezena
   \cadre{8}
\end{tikzpicture}
\end{minipage}

\vspace{0.5cm}
\paragraph{5237+8409+9584}= 23230\\

\vspace{0.2cm}
\begin{minipage}{0.3\textwidth}
\begin{tikzpicture}
   \tige{1}{0}{0}
   \tige{2}{0}{0}       %<- here a dot: unit million
   \tige{3}{5}{1}       %<- here no dot: hundreds of thousands
   \tige{4}{2}{0}       %<- here no dot: tens of thousands
   \tige{5}{3}{0}       %<- here a dot: units thousands
   \tige{6}{7}{1}       %% milhar
   \tige{7}{0}{0}       %% centena
   \tige{8}{0}{0}       %% dezena
   \cadre{8}
\end{tikzpicture}
\end{minipage}%
%
\begin{minipage}{0.3\textwidth}
\begin{tikzpicture}
   \tige{1}{0}{0}
   \tige{2}{1}{0}       %<- here a dot: unit million
   \tige{3}{3}{1}       %<- here no dot: hundreds of thousands
   \tige{4}{6}{0}       %<- here no dot: tens of thousands
   \tige{5}{4}{0}       %<- here a dot: units thousands
   \tige{6}{6}{1}       %% milhar
   \tige{7}{0}{0}       %% centena
   \tige{8}{0}{0}       %% dezena
   \cadre{8}
\end{tikzpicture}
\end{minipage}%
%
\begin{minipage}{0.3\textwidth}
\begin{tikzpicture}
   \tige{1}{0}{0}
   \tige{2}{2}{0}       %<- here a dot: unit million
   \tige{3}{3}{1}       %<- here no dot: hundreds of thousands
   \tige{4}{2}{0}       %<- here no dot: tens of thousands
   \tige{5}{3}{0}       %<- here a dot: units thousands
   \tige{6}{0}{1}       %% milhar
   \tige{7}{0}{0}       %% centena
   \tige{8}{0}{0}       %% dezena
   \cadre{8}
\end{tikzpicture}
\end{minipage}

\section{Soma com centavos}
Como você pode ver abaixo, é muito fácil acrescentar centavos
no esquema de soma. Basta escrever as frações depois
do ponto decimal da direita. Há duas casas depois
desse ponto, justamente para registrar os centavos.

\paragraph{964.5+671.2+1500}= 3135.7\\

\vspace{0.2cm}
\begin{minipage}{0.3\textwidth}
\begin{tikzpicture}
   \tige{1}{0}{0}
   \tige{2}{0}{0}       %<- here a dot: unit million
   \tige{3}{0}{1}       %<- here no dot: hundreds of thousands
   \tige{4}{9}{0}       %<- here no dot: tens of thousands
   \tige{5}{6}{0}       %<- here a dot: units thousands
   \tige{6}{4}{1}       %% milhar
   \tige{7}{5}{0}       %% centena
   \tige{8}{0}{0}       %% dezena
   \cadre{8}
\end{tikzpicture}
\end{minipage}%
%
\begin{minipage}{0.3\textwidth}
\begin{tikzpicture}
   \tige{1}{0}{0}
   \tige{2}{0}{0}       %<- here a dot: unit million
   \tige{3}{1}{1}       %<- here no dot: hundreds of thousands
   \tige{4}{6}{0}       %<- here no dot: tens of thousands
   \tige{5}{3}{0}       %<- here a dot: units thousands
   \tige{6}{5}{1}       %% milhar
   \tige{7}{7}{0}       %% centena
   \tige{8}{0}{0}       %% dezena
   \cadre{8}
\end{tikzpicture}
\end{minipage}%
%
\begin{minipage}{0.3\textwidth}
\begin{tikzpicture}
   \tige{1}{0}{0}
   \tige{2}{0}{0}       %<- here a dot: unit million
   \tige{3}{3}{1}       %<- here no dot: hundreds of thousands
   \tige{4}{1}{0}       %<- here no dot: tens of thousands
   \tige{5}{3}{0}       %<- here a dot: units thousands
   \tige{6}{5}{1}       %% milhar
   \tige{7}{7}{0}       %% centena
   \tige{8}{0}{0}       %% dezena
   \cadre{8}
\end{tikzpicture}
\end{minipage}

\paragraph{964.5+671.2+1598.5}= 3234.2\\

\vspace{0.2cm}
\begin{minipage}{0.3\textwidth}
\begin{tikzpicture}
   \tige{1}{0}{0}
   \tige{2}{0}{0}       %<- here a dot: unit million
   \tige{3}{0}{1}       %<- here no dot: hundreds of thousands
   \tige{4}{9}{0}       %<- here no dot: tens of thousands
   \tige{5}{6}{0}       %<- here a dot: units thousands
   \tige{6}{4}{1}       %% milhar
   \tige{7}{5}{0}       %% centena
   \tige{8}{0}{0}       %% dezena
   \cadre{8}
\end{tikzpicture}
\end{minipage}%
%
\begin{minipage}{0.3\textwidth}
\begin{tikzpicture}
   \tige{1}{0}{0}
   \tige{2}{0}{0}       %<- here a dot: unit million
   \tige{3}{1}{1}       %<- here no dot: hundreds of thousands
   \tige{4}{6}{0}       %<- here no dot: tens of thousands
   \tige{5}{3}{0}       %<- here a dot: units thousands
   \tige{6}{5}{1}       %% milhar
   \tige{7}{7}{0}       %% centena
   \tige{8}{0}{0}       %% dezena
   \cadre{8}
\end{tikzpicture}
\end{minipage}%
%
\begin{minipage}{0.3\textwidth}
\begin{tikzpicture}
   \tige{1}{0}{0}
   \tige{2}{0}{0}       %<- here a dot: unit million
   \tige{3}{3}{1}       %<- here no dot: hundreds of thousands
   \tige{4}{2}{0}       %<- here no dot: tens of thousands
   \tige{5}{3}{0}       %<- here a dot: units thousands
   \tige{6}{4}{1}       %% milhar
   \tige{7}{2}{0}       %% centena
   \tige{8}{0}{0}       %% dezena
   \cadre{8}
\end{tikzpicture}
\end{minipage}


\chapter{Multiplicação}

\paragraph{47 $\times$ 38.}
Colocamos o multiplicando 47 no extremidade
esquerda do soroban. Como o multiplicando
tem 2 algarismos, reservamos 2+1 algarismos
na extremidade direita para o resultado.
Ao lado do resultado, na quinta e quarta colunas
da direita, posicionamos o multiplicador,
que é 38. Agora, multiplicamos as 4 dezenas
de 47 pelas 8 unidades de 38 e colocamos 32 nas centenas
e dezenas do resultado. Em seguida,
multiplicamos as 7 unidades de 47 pelas 8
unidades de 38 e adicionamos 56 às
dezenas e unidades do resultado.\\

\vspace{0.25cm}
\begin{minipage}{0.4\textwidth}
\begin{tikzpicture}
   \tige{1}{4}{0}
   \tige{2}{7}{0}       %<- here a dot: unit million
   \tige{3}{0}{1}       %<- here no dot: hundreds of thousands
   \tige{4}{0}{0}       %<- here no dot: tens of thousands
   \tige{5}{0}{0}       %<- here a dot: units thousands
   \tige{6}{0}{1}       %% milhar
   \tige{7}{3}{0}       %% centena
   \tige{8}{8}{0}       %% dezena
   \tige{9}{0}{1}
   \tige{10}{0}{0}
   \tige{11}{0}{0}
   \binoire{1}{2}{gray}
   \binoire{1}{3}{gray}
   \binoire{1}{4}{gray}
   \binoire{1}{5}{gray}
   \binoire{2}{10}{gray}
   \binoire{2}{5}{gray}
   \binoire{2}{4}{gray}
   \binoire{7}{3}{gray}
   \binoire{7}{4}{gray}
   \binoire{7}{5}{gray}
   \binoire{8}{3}{gray}
   \binoire{8}{4}{gray}
   \binoire{8}{5}{gray}
   \binoire{8}{10}{gray}   
   \cadre{11}
\end{tikzpicture}
\end{minipage}%
%
\begin{minipage}{0.3\textwidth}
\begin{tikzpicture}
   \tige{1}{0}{1}       %% milhar
   \tige{2}{3}{0}       %% centena
   \tige{3}{8}{0}       %% dezena
   \tige{4}{3}{1}
   \tige{5}{2}{0}
   \tige{6}{0}{0}
   \binoire{2}{3}{gray}
   \binoire{2}{4}{gray}
   \binoire{2}{5}{gray}
   \binoire{3}{3}{gray}
   \binoire{3}{4}{gray}
   \binoire{3}{5}{gray}
   \binoire{3}{10}{gray}   
\end{tikzpicture}
\end{minipage}%
%
\begin{minipage}{0.3\textwidth}
\begin{tikzpicture}
   \tige{1}{0}{0}       %<- here a dot: units thousands
   \tige{2}{3}{1}       %% milhar
   \tige{3}{8}{0}       %% dezena
   \tige{4}{3}{1}
   \tige{5}{7}{0}
   \tige{6}{6}{0}
   \binoire{2}{3}{gray}
   \binoire{2}{4}{gray}
   \binoire{2}{5}{gray}
   \binoire{3}{3}{gray}
   \binoire{3}{4}{gray}
   \binoire{3}{5}{gray}
   \binoire{3}{10}{gray}
\end{tikzpicture}
\end{minipage}

\vspace{0.25cm}
Agora que terminamos as 8 unidades do
multiplicador, podemos eliminá-las e
começar a trabalhar com as 3 dezenas.
Multiplicamos as 4 dezenas de 47 por
3 e adicionamos 12 aos milhares e centenas
do resultado. Em seguida multiplicamos
as 7 unidades de 47 por 3 e adicionamos 21
às centenas e dezenas do resultado,
que é 1786.

\vspace{0.25cm}
\begin{minipage}{0.4\textwidth}
\begin{tikzpicture}
   \tige{1}{4}{0}
   \tige{2}{7}{0}       %<- here a dot: unit million
   \tige{3}{0}{1}       %<- here no dot: hundreds of thousands
   \tige{4}{0}{0}       %<- here no dot: tens of thousands
   \tige{5}{0}{0}       %<- here a dot: units thousands
   \tige{6}{0}{1}       %% milhar
   \tige{7}{3}{1}       %% milhar
   \tige{8}{0}{0}       %% dezena
   \tige{9}{3}{1}
   \tige{10}{7}{0}
   \tige{11}{6}{0}
   \cadre{11}
   \binoire{1}{2}{gray}
   \binoire{1}{3}{gray}
   \binoire{1}{4}{gray}
   \binoire{1}{5}{gray}
   \binoire{2}{10}{gray}
   \binoire{2}{5}{gray}
   \binoire{2}{4}{gray}
   \binoire{7}{3}{gray}
   \binoire{7}{4}{gray}
   \binoire{7}{5}{gray}
\end{tikzpicture}
\end{minipage}%
%
\begin{minipage}{0.3\textwidth}
\begin{tikzpicture}
   \tige{1}{0}{1}       %% milhar
   \tige{2}{3}{0}       %% centena
   \tige{3}{1}{0}       %% dezena
   \tige{4}{5}{1}
   \tige{5}{7}{0}
   \tige{6}{6}{0}
   \binoire{2}{3}{gray}
   \binoire{2}{4}{gray}
   \binoire{2}{5}{gray}
\end{tikzpicture}
\end{minipage}%
%
\begin{minipage}{0.3\textwidth}
\begin{tikzpicture}
   \tige{1}{0}{0}       %<- here a dot: units thousands
   \tige{2}{3}{1}       %% milhar
   \tige{3}{1}{0}       %% dezena
   \tige{4}{7}{1}
   \tige{5}{8}{0}
   \tige{6}{6}{0}
   \binoire{2}{3}{gray}
   \binoire{2}{4}{gray}
   \binoire{2}{5}{gray}
\end{tikzpicture}
\end{minipage}

%%% daqui
\paragraph{23 $\times$ 14.}
Posicionamos o multiplicando 23 no extremidade
esquerda do soroban. Considerando que 23
tem 2 dígitos, reservamos 2+1 posições
na extremidade direita para o resultado.
Ao lado do resultado, na quinta e quarta colunas
da direita, pomos a multiplicador 14.
Agora, multiplicamos $2\times 4$ e colocamos
08 nas centenas e dezenas do resultado.
Depois,
multiplicamos $3\times 4$ e somamos 12 às
dezenas e unidades do resultado.\\

\vspace{0.25cm}
\begin{minipage}{0.4\textwidth}
\begin{tikzpicture}
   \tige{1}{2}{0}
   \tige{2}{3}{0}       %<- here a dot: unit million
   \tige{3}{0}{1}       %<- here no dot: hundreds of thousands
   \tige{4}{0}{0}       %<- here no dot: tens of thousands
   \tige{5}{0}{0}       %<- here a dot: units thousands
   \tige{6}{0}{1}       %% milhar
   \tige{7}{1}{0}       %% centena
   \tige{8}{4}{0}       %% dezena
   \tige{9}{0}{1}
   \tige{10}{0}{0}
   \tige{11}{0}{0}
   \binoire{1}{5}{gray}
   \binoire{1}{4}{gray}
   \binoire{2}{5}{gray}
   \binoire{2}{4}{gray}
   \binoire{2}{3}{gray}
   \binoire{7}{5}{gray}
   \binoire{8}{5}{gray}
   \binoire{8}{4}{gray}
   \binoire{8}{3}{gray}
   \binoire{8}{2}{gray}   
   \cadre{11}
\end{tikzpicture}
\end{minipage}%
%
\begin{minipage}{0.3\textwidth}
\begin{tikzpicture}
   \tige{1}{0}{1}       %% milhar
   \tige{2}{1}{0}       %% centena
   \tige{3}{4}{0}       %% dezena
   \tige{4}{0}{1}
   \tige{5}{8}{0}
   \tige{6}{0}{0}
   \binoire{2}{5}{gray}
   \binoire{3}{2}{gray}
   \binoire{3}{3}{gray}
   \binoire{3}{4}{gray}
   \binoire{3}{5}{gray}   
\end{tikzpicture}
\end{minipage}%
%
\begin{minipage}{0.3\textwidth}
\begin{tikzpicture}
   \tige{1}{0}{1}       %% milhar
   \tige{2}{1}{0}       %% centena
   \tige{3}{4}{0}       %% dezena
   \tige{4}{0}{1}
   \tige{5}{9}{0}
   \tige{6}{2}{0}
   \binoire{2}{5}{gray}
   \binoire{3}{2}{gray}
   \binoire{3}{3}{gray}
   \binoire{3}{4}{gray}
   \binoire{3}{5}{gray}   
\end{tikzpicture}
\end{minipage}

\vspace{0.25cm}
Agora que terminamos as 4 unidades do
multiplicador, podemos eliminá-las e
começar a trabalhar com a dezena.
Multiplicamos 2 por
1 e adicionamos 02 aos milhares e centenas
do resultado. Em seguida multiplicamos
as 3 por 1 e adicionamos 03
às centenas e dezenas do resultado
final: 322.

\vspace{0.25cm}
\begin{minipage}{0.4\textwidth}
\begin{tikzpicture}
   \tige{1}{2}{0}
   \tige{2}{3}{0}       %<- here a dot: unit million
   \tige{3}{0}{1}       %<- here no dot: hundreds of thousands
   \tige{4}{0}{0}       %<- here no dot: tens of thousands
   \tige{5}{0}{0}       %<- here a dot: units thousands
   \tige{6}{0}{1}       %% milhar
   \tige{7}{1}{1}       %% milhar
   \tige{8}{0}{0}       %% dezena
   \tige{9}{2}{1}
   \tige{10}{9}{0}
   \tige{11}{2}{0}
   \cadre{11}
   \binoire{1}{4}{gray}
   \binoire{1}{5}{gray}
   \binoire{2}{5}{gray}
   \binoire{2}{4}{gray}
   \binoire{2}{3}{gray}
   \binoire{7}{5}{gray}
\end{tikzpicture}
\end{minipage}%
%
\begin{minipage}{0.3\textwidth}
\begin{tikzpicture}
   \tige{1}{0}{1}       %% milhar
   \tige{2}{1}{0}       %% centena
   \tige{3}{0}{0}       %% dezena
   \tige{4}{2}{1}
   \tige{5}{9}{0}
   \tige{6}{2}{0}
   \binoire{2}{5}{gray}
\end{tikzpicture}
\end{minipage}%
%
\begin{minipage}{0.3\textwidth}
\begin{tikzpicture}
   \tige{1}{0}{0}       %<- here a dot: units thousands
   \tige{2}{1}{1}       %% milhar
   \tige{3}{0}{0}       %% dezena
   \tige{4}{3}{1}
   \tige{5}{2}{0}
   \tige{6}{2}{0}
   \binoire{2}{5}{gray}
\end{tikzpicture}
\end{minipage}

O importante nessa história toda é
que, quando o produto de dois dígitos
tem apenas um algarismo, devemos acrescentar
um zero à esquerda do algarismo.


%%% até aqui






\paragraph{78 $\times$ 86}= $6708$\\

\begin{minipage}{0.4\textwidth}
\begin{tikzpicture}
   \tige{1}{7}{0}
   \tige{2}{8}{0}       %<- here a dot: unit million
   \tige{3}{0}{1}       %<- here no dot: hundreds of thousands
   \tige{4}{0}{0}       %<- here no dot: tens of thousands
   \tige{5}{0}{0}       %<- here a dot: units thousands
   \tige{6}{0}{1}       %% milhar
   \tige{7}{8}{0}       %% centena
   \tige{8}{6}{0}       %% dezena
   \tige{9}{0}{1}
   \tige{10}{0}{0}
   \tige{11}{0}{0}
   \cadre{11}
   \binoire{1}{10}{gray}
   \binoire{1}{5}{gray}
   \binoire{1}{4}{gray}
   \binoire{2}{10}{gray}
   \binoire{2}{5}{gray}
   \binoire{2}{4}{gray}
   \binoire{2}{3}{gray}
   \binoire{7}{10}{gray}
   \binoire{7}{5}{gray}
   \binoire{7}{4}{gray}
   \binoire{7}{3}{gray}
   \binoire{8}{10}{gray}
   \binoire{8}{5}{gray}   
\end{tikzpicture}
\end{minipage}%
%
\begin{minipage}{0.3\textwidth}
\begin{tikzpicture}
   \tige{1}{0}{1}       %% milhar
   \tige{2}{8}{0}       %% centena
   \tige{3}{6}{0}       %% dezena
   \tige{4}{4}{1}
   \tige{5}{2}{0}
   \tige{6}{0}{0}
   \binoire{2}{10}{gray}
   \binoire{2}{5}{gray}
   \binoire{2}{4}{gray}
   \binoire{2}{3}{gray}
   \binoire{3}{10}{gray}
   \binoire{3}{5}{gray}   
\end{tikzpicture}
\end{minipage}%
%
\begin{minipage}{0.3\textwidth}
\begin{tikzpicture}
   \tige{1}{0}{0}       %<- here a dot: units thousands
   \tige{2}{8}{1}       %% milhar
   \tige{3}{6}{0}       %% dezena
   \tige{4}{4}{1}
   \tige{5}{6}{0}
   \tige{6}{8}{0}
   \binoire{2}{10}{gray}
   \binoire{2}{5}{gray}
   \binoire{2}{4}{gray}
   \binoire{2}{3}{gray}
   \binoire{3}{10}{gray}
   \binoire{3}{5}{gray}
\end{tikzpicture}
\end{minipage}


%% 78 * 86
\vspace{0.2cm}
\begin{minipage}{0.4\textwidth}
\begin{tikzpicture}
   \tige{1}{7}{0}
   \tige{2}{8}{0}       %<- here a dot: unit million
   \tige{3}{0}{1}       %<- here no dot: hundreds of thousands
   \tige{4}{0}{0}       %<- here no dot: tens of thousands
   \tige{5}{0}{0}       %<- here a dot: units thousands
   \tige{6}{0}{1}       %% milhar
   \tige{7}{8}{0}       %% centena
   \tige{8}{0}{0}       %% dezena
   \tige{9}{4}{1}
   \tige{10}{6}{0}
   \tige{11}{8}{0}
   \cadre{11}
   \binoire{1}{10}{gray}
   \binoire{1}{5}{gray}
   \binoire{1}{4}{gray}
   \binoire{2}{10}{gray}
   \binoire{2}{5}{gray}
   \binoire{2}{4}{gray}
   \binoire{2}{3}{gray}
   \binoire{7}{10}{gray}
   \binoire{7}{5}{gray}
   \binoire{7}{4}{gray}
   \binoire{7}{3}{gray}
\end{tikzpicture}
\end{minipage}%
%
\begin{minipage}{0.3\textwidth}
\begin{tikzpicture}
   \tige{1}{0}{0}       %<- here a dot: units thousands
   \tige{2}{8}{1}       %% milhar
   \tige{3}{6}{0}       %% dezena
   \tige{4}{0}{1}
   \tige{5}{6}{0}
   \tige{6}{8}{0}
   \binoire{2}{10}{gray}
   \binoire{2}{5}{gray}
   \binoire{2}{4}{gray}
   \binoire{2}{3}{gray}   
\end{tikzpicture}
\end{minipage}%
%
\begin{minipage}{0.3\textwidth}
\begin{tikzpicture}
   \tige{1}{0}{1}       %% milhar
   \tige{2}{8}{0}       %% centena
   \tige{3}{6}{0}       %% dezena
   \tige{4}{7}{1}
   \tige{5}{0}{0}
   \tige{6}{8}{0}
   \binoire{2}{10}{gray}
   \binoire{2}{5}{gray}
   \binoire{2}{4}{gray}
   \binoire{2}{3}{gray}
\end{tikzpicture}
\end{minipage}
%% end 78 * 86

\vspace{0.3cm}
\paragraph{92 $\times$ 67}= $6164$\\

\begin{minipage}{0.4\textwidth}
\begin{tikzpicture}
   \tige{1}{9}{0}
   \tige{2}{2}{0}       %<- here a dot: unit million
   \tige{3}{0}{1}       %<- here no dot: hundreds of thousands
   \tige{4}{0}{0}       %<- here no dot: tens of thousands
   \tige{5}{0}{0}       %<- here a dot: units thousands
   \tige{6}{0}{1}       %% milhar
   \tige{7}{6}{0}       %% centena
   \tige{8}{7}{0}       %% dezena
   \tige{9}{0}{1}
   \tige{10}{0}{0}
   \tige{11}{0}{0}
   \cadre{11}
   \binoire{1}{10}{gray}
   \binoire{1}{5}{gray}
   \binoire{1}{4}{gray}
   \binoire{1}{3}{gray}
   \binoire{1}{2}{gray}
   \binoire{2}{5}{gray}
   \binoire{2}{4}{gray}
   \binoire{7}{10}{gray}
   \binoire{7}{5}{gray}
   \binoire{8}{10}{gray}
   \binoire{8}{5}{gray}
   \binoire{8}{4}{gray}
\end{tikzpicture}
\end{minipage}%
%
\begin{minipage}{0.3\textwidth}
\begin{tikzpicture}
   \tige{1}{0}{1}       %% milhar
   \tige{2}{6}{0}       %% centena
   \tige{3}{7}{0}       %% dezena
   \tige{4}{6}{1}
   \tige{5}{3}{0}
   \tige{6}{0}{0}
   \binoire{2}{10}{gray}
   \binoire{2}{5}{gray}
   \binoire{3}{10}{gray}
   \binoire{3}{5}{gray}
   \binoire{3}{4}{gray}
\end{tikzpicture}
\end{minipage}%
%
\begin{minipage}{0.3\textwidth}
\begin{tikzpicture}
   \tige{1}{0}{1}       %% milhar
   \tige{2}{6}{0}       %% centena
   \tige{3}{7}{0}       %% dezena
   \tige{4}{6}{1}
   \tige{5}{4}{0}
   \tige{6}{4}{0}
   \binoire{2}{10}{gray}
   \binoire{2}{5}{gray}
   \binoire{3}{10}{gray}
   \binoire{3}{5}{gray}
   \binoire{3}{4}{gray}
\end{tikzpicture}
\end{minipage}

%% 92 * 67
\vspace{0.2cm}
\begin{minipage}{0.4\textwidth}
\begin{tikzpicture}
   \tige{1}{9}{0}
   \tige{2}{2}{0}       %<- here a dot: unit million
   \tige{3}{0}{1}       %<- here no dot: hundreds of thousands
   \tige{4}{0}{0}       %<- here no dot: tens of thousands
   \tige{5}{0}{0}       %<- here a dot: units thousands
   \tige{6}{0}{1}       %% milhar
   \tige{7}{6}{0}       %% centena
   \tige{8}{0}{0}       %% dezena
   \tige{9}{6}{1}
   \tige{10}{4}{0}
   \tige{11}{4}{0}
   \cadre{11}
   \binoire{1}{10}{gray}
   \binoire{1}{5}{gray}
   \binoire{1}{4}{gray}
   \binoire{1}{3}{gray}
   \binoire{1}{2}{gray}
   \binoire{2}{5}{gray}
   \binoire{2}{4}{gray}
   \binoire{7}{10}{gray}
   \binoire{7}{5}{gray}
\end{tikzpicture}
\end{minipage}%
%
\begin{minipage}{0.3\textwidth}
\begin{tikzpicture}
   \tige{1}{0}{1}       %% milhar
   \tige{2}{6}{0}       %% centena
   \tige{3}{6}{0}       %% dezena
   \tige{4}{0}{1}
   \tige{5}{4}{0}
   \tige{6}{4}{0}
   \binoire{2}{10}{gray}
   \binoire{2}{5}{gray}
\end{tikzpicture}
\end{minipage}%
%
\begin{minipage}{0.3\textwidth}
\begin{tikzpicture}
   \tige{1}{0}{1}       %% milhar
   \tige{2}{6}{0}       %% centena
   \tige{3}{6}{0}       %% dezena
   \tige{4}{1}{1}
   \tige{5}{6}{0}
   \tige{6}{4}{0}
   \binoire{2}{10}{gray}
   \binoire{2}{5}{gray}
\end{tikzpicture}
\end{minipage}
%% end 92 * 67


\vspace{0.5cm}
\paragraph{42 $\times$ 56}= $2352$\\

\begin{minipage}{0.4\textwidth}
\begin{tikzpicture}
   \tige{1}{4}{0}
   \tige{2}{2}{0}       %<- here a dot: unit million
   \tige{3}{0}{1}       %<- here no dot: hundreds of thousands
   \tige{4}{0}{0}       %<- here no dot: tens of thousands
   \tige{5}{0}{0}       %<- here a dot: units thousands
   \tige{6}{0}{1}       %% milhar
   \tige{7}{5}{0}       %% centena
   \tige{8}{6}{0}       %% dezena
   \tige{9}{0}{1}
   \tige{10}{0}{0}
   \tige{11}{0}{0}
   \cadre{11}
   \binoire{1}{5}{gray}
   \binoire{1}{4}{gray}
   \binoire{1}{3}{gray}
   \binoire{1}{2}{gray}
   \binoire{2}{5}{gray}
   \binoire{2}{4}{gray}
   \binoire{7}{10}{gray}
   \binoire{8}{10}{gray}
   \binoire{8}{5}{gray}
\end{tikzpicture}
\end{minipage}%
%
\begin{minipage}{0.3\textwidth}
\begin{tikzpicture}
   \tige{1}{0}{1}       %% milhar
   \tige{2}{5}{0}       %% centena
   \tige{3}{6}{0}       %% dezena
   \tige{4}{2}{1}
   \tige{5}{4}{0}
   \tige{6}{0}{0}
   \binoire{2}{10}{gray}
   \binoire{3}{10}{gray}
   \binoire{3}{5}{gray}
\end{tikzpicture}
\end{minipage}%
%
\begin{minipage}{0.3\textwidth}
\begin{tikzpicture}
   \tige{1}{0}{0}       %<- here a dot: units thousands
   \tige{2}{5}{1}       %% milhar
   \tige{3}{6}{0}       %% dezena
   \tige{4}{2}{1}
   \tige{5}{5}{0}
   \tige{6}{2}{0}
   \binoire{2}{10}{gray}
   \binoire{3}{10}{gray}
   \binoire{3}{5}{gray}
\end{tikzpicture}
\end{minipage}

%% 42 * 56
\vspace{0.2cm}
\begin{minipage}{0.4\textwidth}
\begin{tikzpicture}
   \tige{1}{4}{0}
   \tige{2}{2}{0}       %<- here a dot: unit million
   \tige{3}{0}{1}       %<- here no dot: hundreds of thousands
   \tige{4}{0}{0}       %<- here no dot: tens of thousands
   \tige{5}{0}{0}       %<- here a dot: units thousands
   \tige{6}{0}{1}       %% milhar
   \tige{7}{5}{0}       %% centena
   \tige{8}{0}{0}       %% dezena
   \tige{9}{2}{1}
   \tige{10}{5}{0}
   \tige{11}{2}{0}
   \cadre{11}
   \binoire{1}{5}{gray}
   \binoire{1}{4}{gray}
   \binoire{1}{3}{gray}
   \binoire{1}{2}{gray}
   \binoire{2}{5}{gray}
   \binoire{2}{4}{gray}
   \binoire{7}{10}{gray}
\end{tikzpicture}
\end{minipage}%
%
\begin{minipage}{0.3\textwidth}
\begin{tikzpicture}
   \tige{1}{0}{1}       %% milhar
   \tige{2}{5}{0}       %% centena
   \tige{3}{2}{0}       %% dezena
   \tige{4}{2}{1}
   \tige{5}{5}{0}
   \tige{6}{2}{0}
   \binoire{2}{10}{gray}
\end{tikzpicture}
\end{minipage}%
%
\begin{minipage}{0.3\textwidth}
\begin{tikzpicture}
   \tige{1}{0}{0}       %<- here a dot: units thousands
   \tige{2}{5}{1}       %% milhar
   \tige{3}{2}{0}       %% dezena
   \tige{4}{3}{1}
   \tige{5}{5}{0}
   \tige{6}{2}{0}
   \binoire{2}{10}{gray}
\end{tikzpicture}
\end{minipage}
%% end 42 * 56


\vspace{0.8cm}
\paragraph{68 $\times$ 85}= $5780$\\

\begin{minipage}{0.4\textwidth}
\begin{tikzpicture}
   \tige{1}{6}{0}
   \tige{2}{8}{0}       %<- here a dot: unit million
   \tige{3}{0}{1}       %<- here no dot: hundreds of thousands
   \tige{4}{0}{0}       %<- here no dot: tens of thousands
   \tige{5}{0}{0}       %<- here a dot: units thousands
   \tige{6}{0}{1}       %% milhar
   \tige{7}{8}{0}       %% centena
   \tige{8}{5}{0}       %% dezena
   \tige{9}{0}{1}
   \tige{10}{0}{0}
   \tige{11}{0}{0}
   \cadre{11}
   \binoire{1}{10}{gray}
   \binoire{1}{5}{gray}
   \binoire{2}{10}{gray}
   \binoire{2}{5}{gray}
   \binoire{2}{4}{gray}
   \binoire{2}{3}{gray}
   \binoire{7}{10}{gray}
   \binoire{7}{5}{gray}
   \binoire{7}{4}{gray}
   \binoire{7}{3}{gray}
   \binoire{8}{10}{gray}
\end{tikzpicture}
\end{minipage}%
%
\begin{minipage}{0.3\textwidth}
\begin{tikzpicture}
   \tige{1}{0}{1}       %% milhar
   \tige{2}{8}{0}       %% centena
   \tige{3}{5}{0}       %% dezena
   \tige{4}{3}{1}
   \tige{5}{0}{0}
   \tige{6}{0}{0}
   \binoire{2}{10}{gray}
   \binoire{2}{5}{gray}
   \binoire{2}{4}{gray}
   \binoire{2}{3}{gray}
   \binoire{3}{10}{gray}
\end{tikzpicture}
\end{minipage}%
%
\begin{minipage}{0.3\textwidth}
\begin{tikzpicture}
   \tige{1}{0}{0}       %<- here a dot: units thousands
   \tige{2}{8}{1}       %% milhar
   \tige{3}{5}{0}       %% dezena
   \tige{4}{3}{1}
   \tige{5}{4}{0}
   \tige{6}{0}{0}
      \binoire{2}{10}{gray}
   \binoire{2}{5}{gray}
   \binoire{2}{4}{gray}
   \binoire{2}{3}{gray}
   \binoire{3}{10}{gray}
\end{tikzpicture}
\end{minipage}

%% 68 * 85
\vspace{0.2cm}
\begin{minipage}{0.4\textwidth}
\begin{tikzpicture}
   \tige{1}{6}{0}
   \tige{2}{8}{0}       %<- here a dot: unit million
   \tige{3}{0}{1}       %<- here no dot: hundreds of thousands
   \tige{4}{0}{0}       %<- here no dot: tens of thousands
   \tige{5}{0}{0}       %<- here a dot: units thousands
   \tige{6}{0}{1}       %% milhar
   \tige{7}{8}{0}       %% centena
   \tige{8}{0}{0}       %% dezena
   \tige{9}{3}{1}
   \tige{10}{4}{0}
   \tige{11}{0}{0}
   \cadre{11}
   \binoire{1}{10}{gray}
   \binoire{1}{5}{gray}
   \binoire{2}{10}{gray}
   \binoire{2}{5}{gray}
   \binoire{2}{4}{gray}
   \binoire{2}{3}{gray}
   \binoire{7}{10}{gray}
   \binoire{7}{5}{gray}
   \binoire{7}{4}{gray}
   \binoire{7}{3}{gray}
\end{tikzpicture}
\end{minipage}%
%
\begin{minipage}{0.3\textwidth}
\begin{tikzpicture}
   \tige{1}{0}{1}       %% milhar
   \tige{2}{8}{0}       %% centena
   \tige{3}{5}{0}       %% dezena
   \tige{4}{1}{1}
   \tige{5}{4}{0}
   \tige{6}{0}{0}
      \binoire{2}{10}{gray}
   \binoire{2}{5}{gray}
   \binoire{2}{4}{gray}
   \binoire{2}{3}{gray}
\end{tikzpicture}
\end{minipage}%
%
\begin{minipage}{0.3\textwidth}
\begin{tikzpicture}
   \tige{1}{0}{0}       %<- here a dot: units thousands
   \tige{2}{8}{1}       %% milhar
   \tige{3}{5}{0}       %% dezena
   \tige{4}{7}{1}
   \tige{5}{8}{0}
   \tige{6}{0}{0}
      \binoire{2}{10}{gray}
   \binoire{2}{5}{gray}
   \binoire{2}{4}{gray}
   \binoire{2}{3}{gray}
\end{tikzpicture}
\end{minipage}
%% end 68 * 85




\paragraph{27 $\times$ 96}= $2592$\\

\begin{minipage}{0.4\textwidth}
\begin{tikzpicture}
   \tige{1}{2}{0}
   \tige{2}{7}{0}       %<- here a dot: unit million
   \tige{3}{0}{1}       %<- here no dot: hundreds of thousands
   \tige{4}{0}{0}       %<- here no dot: tens of thousands
   \tige{5}{0}{0}       %<- here a dot: units thousands
   \tige{6}{0}{1}       %% milhar
   \tige{7}{9}{0}       %% centena
   \tige{8}{6}{0}       %% dezena
   \tige{9}{0}{1}
   \tige{10}{0}{0}
   \tige{11}{0}{0}
   \cadre{11}
   \binoire{1}{5}{gray}
   \binoire{1}{4}{gray}
   \binoire{2}{10}{gray}
   \binoire{2}{5}{gray}
   \binoire{2}{4}{gray}
   \binoire{7}{10}{gray}
   \binoire{7}{5}{gray}
   \binoire{7}{4}{gray}
   \binoire{7}{3}{gray}
   \binoire{7}{2}{gray}
   \binoire{8}{10}{gray}
   \binoire{8}{5}{gray}
\end{tikzpicture}
\end{minipage}%
%
\begin{minipage}{0.3\textwidth}
\begin{tikzpicture}
   \tige{1}{0}{1}       %% milhar
   \tige{2}{9}{0}       %% centena
   \tige{3}{6}{0}       %% dezena
   \tige{4}{1}{1}
   \tige{5}{2}{0}
   \tige{6}{0}{0}
   \binoire{2}{10}{gray}
   \binoire{2}{5}{gray}
   \binoire{2}{4}{gray}
   \binoire{2}{3}{gray}
   \binoire{2}{2}{gray}
   \binoire{3}{10}{gray}
   \binoire{3}{5}{gray}
\end{tikzpicture}
\end{minipage}%
%
\begin{minipage}{0.3\textwidth}
\begin{tikzpicture}
   \tige{1}{0}{0}       %<- here a dot: units thousands
   \tige{2}{9}{1}       %% milhar
   \tige{3}{6}{0}       %% dezena
   \tige{4}{1}{1}
   \tige{5}{6}{0}
   \tige{6}{2}{0}
   \binoire{2}{10}{gray}
   \binoire{2}{5}{gray}
   \binoire{2}{4}{gray}
   \binoire{2}{3}{gray}
   \binoire{2}{2}{gray}
   \binoire{3}{10}{gray}
   \binoire{3}{5}{gray}
\end{tikzpicture}
\end{minipage}

%% 27 * 96
\vspace{0.5cm}
\begin{minipage}{0.4\textwidth}
\begin{tikzpicture}
   \tige{1}{2}{0}
   \tige{2}{7}{0}       %<- here a dot: unit million
   \tige{3}{0}{1}       %<- here no dot: hundreds of thousands
   \tige{4}{0}{0}       %<- here no dot: tens of thousands
   \tige{5}{0}{0}       %<- here a dot: units thousands
   \tige{6}{0}{1}       %% milhar
   \tige{7}{9}{0}       %% centena
   \tige{8}{0}{0}       %% dezena
   \tige{9}{1}{1}
   \tige{10}{6}{0}
   \tige{11}{2}{0}
   \cadre{11}
   \binoire{1}{5}{gray}
   \binoire{1}{4}{gray}
   \binoire{2}{10}{gray}
   \binoire{2}{5}{gray}
   \binoire{2}{4}{gray}
   \binoire{7}{10}{gray}
   \binoire{7}{5}{gray}
   \binoire{7}{4}{gray}
   \binoire{7}{3}{gray}
   \binoire{7}{2}{gray}
\end{tikzpicture}
\end{minipage}%
%
\begin{minipage}{0.3\textwidth}
\begin{tikzpicture}
   \tige{1}{0}{1}       %% milhar
   \tige{2}{9}{0}       %% centena
   \tige{3}{1}{0}       %% dezena
   \tige{4}{9}{1}
   \tige{5}{6}{0}
   \tige{6}{2}{0}
   \binoire{2}{10}{gray}
   \binoire{2}{5}{gray}
   \binoire{2}{4}{gray}
   \binoire{2}{3}{gray}
   \binoire{2}{2}{gray}
\end{tikzpicture}
\end{minipage}%
%
\begin{minipage}{0.3\textwidth}
\begin{tikzpicture}
   \tige{1}{0}{0}       %<- here a dot: units thousands
   \tige{2}{9}{1}       %% milhar
   \tige{3}{2}{0}       %% dezena
   \tige{4}{5}{1}
   \tige{5}{9}{0}
   \tige{6}{2}{0}
   \binoire{2}{10}{gray}
   \binoire{2}{5}{gray}
   \binoire{2}{4}{gray}
   \binoire{2}{3}{gray}
   \binoire{2}{2}{gray}
\end{tikzpicture}
\end{minipage}
%%end 27 * 96

\vspace{1cm}
\paragraph{37 $\times$ 96}= $3552$\\

\begin{minipage}{0.4\textwidth}
\begin{tikzpicture}
   \tige{1}{3}{0}
   \tige{2}{7}{0}       %<- here a dot: unit million
   \tige{3}{0}{1}       %<- here no dot: hundreds of thousands
   \tige{4}{0}{0}       %<- here no dot: tens of thousands
   \tige{5}{0}{0}       %<- here a dot: units thousands
   \tige{6}{0}{1}       %% milhar
   \tige{7}{9}{0}       %% centena
   \tige{8}{6}{0}       %% dezena
   \tige{9}{0}{1}
   \tige{10}{0}{0}
   \tige{11}{0}{0}
   \cadre{11}
   \binoire{1}{5}{gray}
   \binoire{1}{4}{gray}
   \binoire{1}{3}{gray}
   \binoire{2}{10}{gray}
   \binoire{2}{5}{gray}
   \binoire{2}{4}{gray}
   \binoire{7}{10}{gray}
   \binoire{7}{5}{gray}
   \binoire{7}{4}{gray}
   \binoire{7}{3}{gray}
   \binoire{7}{2}{gray}
   \binoire{8}{10}{gray}
   \binoire{8}{5}{gray}
\end{tikzpicture}
\end{minipage}%
%
\begin{minipage}{0.3\textwidth}
\begin{tikzpicture}
   \tige{1}{0}{1}       %% milhar
   \tige{2}{9}{0}       %% centena
   \tige{3}{6}{0}       %% dezena
   \tige{4}{1}{1}
   \tige{5}{8}{0}
   \tige{6}{0}{0}
   \binoire{2}{10}{gray}
   \binoire{2}{5}{gray}
   \binoire{2}{4}{gray}
   \binoire{2}{3}{gray}
   \binoire{2}{2}{gray}
   \binoire{3}{10}{gray}
   \binoire{3}{5}{gray}
\end{tikzpicture}
\end{minipage}%
%
\begin{minipage}{0.3\textwidth}
\begin{tikzpicture}
   \tige{1}{0}{0}       %<- here a dot: units thousands
   \tige{2}{9}{1}       %% milhar
   \tige{3}{6}{0}       %% dezena
   \tige{4}{2}{1}
   \tige{5}{2}{0}
   \tige{6}{2}{0}
   \binoire{2}{10}{gray}
   \binoire{2}{5}{gray}
   \binoire{2}{4}{gray}
   \binoire{2}{3}{gray}
   \binoire{2}{2}{gray}
   \binoire{3}{10}{gray}
   \binoire{3}{5}{gray}
\end{tikzpicture}
\end{minipage}

%% 37 * 96
\vspace{0.5cm}
\begin{minipage}{0.4\textwidth}
\begin{tikzpicture}
   \tige{1}{3}{0}
   \tige{2}{7}{0}       %<- here a dot: unit million
   \tige{3}{0}{1}       %<- here no dot: hundreds of thousands
   \tige{4}{0}{0}       %<- here no dot: tens of thousands
   \tige{5}{0}{0}       %<- here a dot: units thousands
   \tige{6}{0}{1}       %% milhar
   \tige{7}{9}{0}       %% centena
   \tige{8}{0}{0}       %% dezena
   \tige{9}{2}{1}
   \tige{10}{2}{0}
   \tige{11}{2}{0}
   \cadre{11}
   \binoire{1}{5}{gray}
   \binoire{1}{4}{gray}
   \binoire{1}{3}{gray}
   \binoire{2}{10}{gray}
   \binoire{2}{5}{gray}
   \binoire{2}{4}{gray}
   \binoire{7}{10}{gray}
   \binoire{7}{5}{gray}
   \binoire{7}{4}{gray}
   \binoire{7}{3}{gray}
   \binoire{7}{2}{gray}
\end{tikzpicture}
\end{minipage}%
%
\begin{minipage}{0.3\textwidth}
\begin{tikzpicture}
   \tige{1}{0}{1}       %% milhar
   \tige{2}{9}{0}       %% centena
   \tige{3}{2}{0}       %% dezena
   \tige{4}{9}{1}
   \tige{5}{2}{0}
   \tige{6}{2}{0}
   \binoire{2}{10}{gray}
   \binoire{2}{5}{gray}
   \binoire{2}{4}{gray}
   \binoire{2}{3}{gray}
   \binoire{2}{2}{gray}
\end{tikzpicture}
\end{minipage}%
%
\begin{minipage}{0.3\textwidth}
\begin{tikzpicture}
   \tige{1}{0}{0}       %<- here a dot: units thousands
   \tige{2}{9}{1}       %% milhar
   \tige{3}{3}{0}       %% dezena
   \tige{4}{5}{1}
   \tige{5}{5}{0}
   \tige{6}{2}{0}
   \binoire{2}{10}{gray}
   \binoire{2}{5}{gray}
   \binoire{2}{4}{gray}
   \binoire{2}{3}{gray}
   \binoire{2}{2}{gray}
\end{tikzpicture}
\end{minipage}
%%end 37 * 96


\vspace{1cm}
\paragraph{47 $\times$ 96}= $4512$\\

\begin{minipage}{0.4\textwidth}
\begin{tikzpicture}
   \tige{1}{4}{0}
   \tige{2}{7}{0}       %<- here a dot: unit million
   \tige{3}{0}{1}       %<- here no dot: hundreds of thousands
   \tige{4}{0}{0}       %<- here no dot: tens of thousands
   \tige{5}{0}{0}       %<- here a dot: units thousands
   \tige{6}{0}{1}       %% milhar
   \tige{7}{9}{0}       %% centena
   \tige{8}{6}{0}       %% dezena
   \tige{9}{0}{1}
   \tige{10}{0}{0}
   \tige{11}{0}{0}
   \cadre{11}
   \binoire{1}{5}{gray}
   \binoire{1}{4}{gray}
   \binoire{1}{3}{gray}
   \binoire{1}{2}{gray}
   \binoire{2}{10}{gray}
   \binoire{2}{5}{gray}
   \binoire{2}{4}{gray}
   \binoire{7}{10}{gray}
   \binoire{7}{5}{gray}
   \binoire{7}{4}{gray}
   \binoire{7}{3}{gray}
   \binoire{7}{2}{gray}
   \binoire{8}{10}{gray}
   \binoire{8}{5}{gray}
\end{tikzpicture}
\end{minipage}%
%
\begin{minipage}{0.3\textwidth}
\begin{tikzpicture}
   \tige{1}{0}{1}       %% milhar
   \tige{2}{9}{0}       %% centena
   \tige{3}{6}{0}       %% dezena
   \tige{4}{2}{1}
   \tige{5}{4}{0}
   \tige{6}{0}{0}
   \binoire{2}{10}{gray}
   \binoire{2}{5}{gray}
   \binoire{2}{4}{gray}
   \binoire{2}{3}{gray}
   \binoire{2}{2}{gray}
   \binoire{3}{10}{gray}
   \binoire{3}{5}{gray}
\end{tikzpicture}
\end{minipage}%
%
\begin{minipage}{0.3\textwidth}
\begin{tikzpicture}
   \tige{1}{0}{0}       %<- here a dot: units thousands
   \tige{2}{9}{1}       %% milhar
   \tige{3}{6}{0}       %% dezena
   \tige{4}{2}{1}
   \tige{5}{8}{0}
   \tige{6}{2}{0}
   \binoire{2}{10}{gray}
   \binoire{2}{5}{gray}
   \binoire{2}{4}{gray}
   \binoire{2}{3}{gray}
   \binoire{2}{2}{gray}
   \binoire{3}{10}{gray}
   \binoire{3}{5}{gray}
\end{tikzpicture}
\end{minipage}

%% 47 * 96
\vspace{0.5cm}
\begin{minipage}{0.4\textwidth}
\begin{tikzpicture}
   \tige{1}{4}{0}
   \tige{2}{7}{0}       %<- here a dot: unit million
   \tige{3}{0}{1}       %<- here no dot: hundreds of thousands
   \tige{4}{2}{0}       %<- here no dot: tens of thousands
   \tige{5}{8}{0}       %<- here a dot: units thousands
   \tige{6}{2}{1}       %% milhar
   \tige{7}{9}{0}       %% centena
   \tige{8}{0}{0}       %% dezena
   \tige{9}{2}{1}
   \tige{10}{8}{0}
   \tige{11}{2}{0}
   \cadre{11}
   \binoire{1}{5}{gray}
   \binoire{1}{4}{gray}
   \binoire{1}{3}{gray}
   \binoire{1}{2}{gray}
   \binoire{2}{10}{gray}
   \binoire{2}{5}{gray}
   \binoire{2}{4}{gray}
   \binoire{7}{10}{gray}
   \binoire{7}{5}{gray}
   \binoire{7}{4}{gray}
   \binoire{7}{3}{gray}
   \binoire{7}{2}{gray}
\end{tikzpicture}
\end{minipage}%
%
\begin{minipage}{0.3\textwidth}
\begin{tikzpicture}
   \tige{1}{0}{1}       %% milhar
   \tige{2}{9}{0}       %% centena
   \tige{3}{3}{0}       %% dezena
   \tige{4}{8}{1}
   \tige{5}{8}{0}
   \tige{6}{2}{0}
   \binoire{2}{10}{gray}
   \binoire{2}{5}{gray}
   \binoire{2}{4}{gray}
   \binoire{2}{3}{gray}
   \binoire{2}{2}{gray}
\end{tikzpicture}
\end{minipage}%
%
\begin{minipage}{0.3\textwidth}
\begin{tikzpicture}
   \tige{1}{0}{0}       %<- here a dot: units thousands
   \tige{2}{9}{1}       %% milhar
   \tige{3}{4}{0}       %% dezena
   \tige{4}{5}{1}
   \tige{5}{1}{0}
   \tige{6}{2}{0}
   \binoire{2}{10}{gray}
   \binoire{2}{5}{gray}
   \binoire{2}{4}{gray}
   \binoire{2}{3}{gray}
   \binoire{2}{2}{gray}
\end{tikzpicture}
\end{minipage}
%%end 47 * 96


\vspace{1cm}
\paragraph{8 $\times$ 96}= $768$\\

\begin{minipage}{0.4\textwidth}
\begin{tikzpicture}
   \tige{1}{8}{0}
   \tige{2}{0}{0}       %<- here a dot: unit million
   \tige{3}{0}{1}       %<- here no dot: hundreds of thousands
   \tige{4}{0}{0}       %<- here no dot: tens of thousands
   \tige{5}{0}{0}       %<- here a dot: units thousands
   \tige{6}{0}{1}       %% milhar
   \tige{7}{0}{0}       %% centena
   \tige{8}{9}{0}       %% dezena
   \tige{9}{6}{1}
   \tige{10}{0}{0}
   \tige{11}{0}{0}
   \cadre{11}
   \binoire{1}{10}{gray}
   \binoire{1}{5}{gray}
   \binoire{1}{4}{gray}
   \binoire{1}{3}{gray}
   \binoire{8}{10}{gray}
   \binoire{8}{5}{gray}
   \binoire{8}{4}{gray}
   \binoire{8}{3}{gray}
   \binoire{8}{2}{gray}
   \binoire{9}{10}{gray}
   \binoire{9}{5}{gray}
\end{tikzpicture}
\end{minipage}%
%
\begin{minipage}{0.3\textwidth}
\begin{tikzpicture}
   \tige{1}{0}{1}       %% milhar
   \tige{2}{0}{0}       %% centena
   \tige{3}{9}{0}       %% dezena
   \tige{4}{6}{1}
   \tige{5}{4}{0}
   \tige{6}{8}{0}
   \binoire{3}{10}{gray}
   \binoire{3}{5}{gray}
   \binoire{3}{4}{gray}
   \binoire{3}{3}{gray}
   \binoire{3}{2}{gray}
   \binoire{4}{10}{gray}
   \binoire{4}{5}{gray}
\end{tikzpicture}
\end{minipage}%
%
\begin{minipage}{0.3\textwidth}
\begin{tikzpicture}
   \tige{1}{0}{0}       %<- here a dot: units thousands
   \tige{2}{0}{1}       %% milhar
   \tige{3}{9}{0}       %% dezena
   \tige{4}{7}{1}
   \tige{5}{6}{0}
   \tige{6}{8}{0}
   \binoire{3}{10}{gray}
   \binoire{3}{5}{gray}
   \binoire{3}{4}{gray}
   \binoire{3}{3}{gray}
   \binoire{3}{2}{gray}
\end{tikzpicture}
\end{minipage}



\section{Taboada}
A primeira coisa que precisamos fazer antes de
mergulhar na matemática é estabelecer um plano
de treinamento para nossa memória. Um bom
exercício para a memória é aprender a toboada
de multiplicação, que daqui para diante chamaremos
apenas de {\em taboada}. 

Sabemos que o conjunto dos seres humanos
é dividida em dois grandes subconjuntos, a saber:
\begin{enumerate}
\item Aqueles que sabem a taboada. Se você é um
dos que sabem a taboada, então aproveite esse
capítulo para firmar ainda mais esse conhecimento.
\item Pessoas que não sabem a taboada. Essas
pessoas não conseguem fazer multiplicação ou divisão
sem calculadora. Se você se encontra
nessa difícil situação, ainda está em tempo de
aprender a taboada. Poékhali! Então vamos lá!
\end{enumerate}

\section{Truques para aprender a taboada}
Evidentemente, você deve reagir quase instantaneamente
quando precisar, por exemplo, do valor de $9\times 6$.
Qualquer hesitação ou utilização de truques pode ter
uma das seguintes consequências em um {\em high stakes test}:
\begin{enumerate}
\item Erros. Hesitação significa que a taboada não está
bem firme no seu cérebro. Nesse caso, você pode utilizar
um valor errado.
\item Demora. Enquanto você está tentando descobrir quanto
vale $9\times 6$, seu concorrente já estará terminando
a próxima pergunta.
\item Cansaço mental. Essa é a pior consequência de não
decorar bem a tabela de multiplicação. Fazer contas sem
domínio total da taboada é muito cansativo.
\end{enumerate}

\begin{tabular}{p{1cm}p{1cm}p{1cm}p{1cm}p{1cm}p{1cm}p{1cm}p{1cm}p{1cm}}
\hline
x & 2  & 3  & 4  & 5  & 6  & 7  & 8  & 9  \\
\hline
2 & 4  & 6  & 8  & 10 & 12   & 14    & 16 & 18 \\
\hline
3 & 6   & 9   & 12   & 15   & 18   & 21   & 24   & 27   \\
\hline
4 & 8   & 12   & 16   & 20   & 24   & 28   & 32   & 36   \\
\hline
5 & 10   & 15   & 20   & 25   & 30   & 35   & 40   & 45   \\
\hline
6 & 12   & 18   & 24   & 30   & 36   & 42   & 48   & 54   \\
\hline
7 & 14   & 21   & 28   & 35   & 42   & 49   & 56   & 63   \\
\hline
8 & 16   & 24   & 32   & 40   & 48   & 56   & 64   & 72   \\
\hline
9 & 18   & 27   & 36   & 45   & 54   & 63   & 72   & 81   \\
\hline
\end{tabular}


\paragraph{Taboada do 9.} Para multiplicar um número $n$ por
nove, basta multiplicar por 10 e subtrair $n$. Seja o exemplo
de $6\times 9$, com o qual começamos essa discussão. Temos
que $6\times 10= 60$. Subtraindo $6$ de $60$,
temos $60-6= 54= 6\times 9$. Você pode utilizar esse
truque no começo, mas não deixe de fazer muitos exercícios,
até conseguir multiplicar qualquer número de 2 a 9 por
9 sem nenhum cansaço mental. Cansaço mental acaba com
o candidato em um {\em high stakes test}.

\paragraph{Taboada do 8.} A taboada do oito é dividida
em duas partes: números de 2 a 5 e números de 6 a 9.

Para um número $n$ de 2 a 5, a dezena de $8\times n$
é dado por $n-1$ e a unidade é dada por $2\times (5-n)$.
Vejamos alguns exemplos:\\

\begin{tabular}{p{2cm}p{2cm}p{4cm}p{4cm}}
   &  & Dezena & Unidade\\
$8\times 2$ &16 & $2-1= 1$ & $2\times(5-2)=6$\\
$8\times 3$ &24 & $3-1= 2$ & $2\times(5-3)=4$\\
$8\times 4$ &32 & $4-1= 3$ & $2\times(5-4)=2$\\
$8\times 5$ &40 & $5-1= 4$ & $2\times(5-5)=0$\\
\end{tabular}

\verb||\\
Quando $n$ vai de 6 a 9, a dezena de $8\times n$ é
dado por $n-2$, e a unidade vale $2\times (10-n)$.\\

\begin{tabular}{p{2cm}p{2cm}p{4cm}p{4cm}}
   &  & Dezena & Unidade\\
$8\times 6$ & 48 & $6-2= 4$ & $2\times(10-6)=8$\\
$8\times 7$ &56 & $7-2= 5$ & $2\times(10-7)=6$\\
$8\times 8$ &64 & $8-2= 6$ & $2\times(10-8)=4$\\
$8\times 9$ &72 & $9-2= 7$ & $2\times(10-9)=2$\\
\end{tabular}


\paragraph{Taboada de 6.} Há três casos a considerar para
encontrar as unidades da taboada de seis
de um número $n$ que vai de 2 a 9.
\begin{enumerate}
\item $n$ é par. Nesse caso, a unidade do produto por 6 é igual a $n$.
\item $n$ é ímpar menor que 5. Nesse caso, a unidade é $n+5$.
\item $n$ é ímpar maior ou igual a 5. Aqui, a unidade é $n-5$.
\end{enumerate}

As dezenas são mais difíceis de se obter. As regras são:
\begin{enumerate}
\item Se o número $n$ é par, a dezena é a metade de $n$.
Consideremos os casos de 2, 4, 6 e 8. 
\begin{itemize}
\item$6\times 2$. Dezena= 2/1= 1, unidade= 2. Resultado: 12.
\item$6\times 4$. Dezena= 4/2= 2, unidade= 4. Resultado: 24.
\item$6\times 6$. Dezena= 6/3= 3, unidade= 6. Resultado: 36.
\item$6\times 8$. Dezena= 8/4= 4, unidade= 8. Resultado: 48.
\end{itemize}
\item No caso de números menores que 5, a dezena é o quociente
da divisão por 2. Apenas o número 3 satisfaz essa condição.
Então, o quociente da divisão de 3 por 2 é 1. Isso significa
que $6\times 3$ é 18.
\item No caso de números ímpares maiores que 5, a dezena é
o quociente da divisão por 2 mais 1. Nesse caso, temos 5,
7 e 9. No caso de 5, a dezena é 3 e $6\times 5= 30$. 
No caso de 7, a dezena é 4, e $6\times 7= 42$.
No caso de 9, a dezena é 5, e $6\times 9= 54$. 
\end{enumerate}

\paragraph{Taboada de 5.} Nesse caso, as unidades são
muito fáceis de encontrar: elas são 5 para $n$ ímpar
e 0 para $n$ par. As dezenas, para $n$ entre 2 e 9,
são obtidas pelo quociente da divisão de $n$ por 2.
Seja o caso de $n=6$. Como 6 é par, a unidade é 0.
A dezena é 6/2= 3. Então $5\times 6= 30$. 
Vejamos outro exemplo de multiplicação de $n$ entre
2 e 9 por 5. No caso do número 7, que é ímpar, a
unidade é 5. A dezena é o quocienet de 7 por 2,
ou seja, 3. Então, o resultado é 35.

Como a taboada de 5 é tão fácil, talvez uma boa regra
para encontrar $6\times n$, com $n$ entre 2 e 9, seja
fazer a conta $5\times n + n$. Por exemplo, seja
encontrar $6\times 7$. Basta encontrar $5\times 7 + 7$,
que dá 42, é claro.

\paragraph{Taboada de 2.} A maneira mais fácil de 
obter a taboada de dois para $n$ entre 2 e 9 é somar
$n$ com ele mesmo. Por exemplo, $2\times 8= 8+8= 16$.

\paragraph{Taboada de 7.}
Sabendo a taboada de 2 e de 5, podemos obter facilmente
a taboada de 7 para $n$ entre 2 e 9. Para isso, 
basta somar $5\times n + 2\times n$. Por exemplo,
para encontrar $7\times 8$,
fazemos $5\times 8 + 2\times 8= 40+16= 56$.
\begin{itemize}
\item $2\times 7= 2\times 5 + 2\times 2= 14$
\item $3\times 7= 3\times 5 + 2\times 3= 21$
\item $4\times 7= 4\times 5 + 2\times 4= 28$
\item $5\times 7= 5\times 5 + 2\times 5= 35$
\item $6\times 7= 6\times 5 + 2\times 6= 42$
\item $7\times 7= 7\times 5 + 2\times 7= 49$
\item $8\times 7= 8\times 5 + 2\times 8= 56$
\item $9\times 7= 9\times 5 + 2\times 9= 63$
\end{itemize}


\paragraph{Taboada de 3.} Para achar $3\times n$,
use $3\times n= 2\times n + n$.
E.g. $3\times 8= 16+8= 24$.

\paragraph{Taboada de 4.}
Para calcular $4\times n$,  basta fazer
$4\times n= 2\times n + 2\times n$.

\chapter{Mais multiplicações}

Considerando que as pedras do ábaco não
podem mudar de cor, não vamos mais marcar
com pedras cinzas o multiplicando e o
multiplicador.\\ 

\vspace{0.3cm}
\paragraph{256 $\times$ 304}= 77824\\
\ladj{0.35}

\vspace{0.2cm}
\begin{minipage}{0.5\textwidth}
\begin{tikzpicture}
   \tige{1}{2}{0}
   \tige{2}{5}{0}       %<- here a dot: unit million
   \tige{3}{6}{1}       %<- here no dot: hundreds of thousands
   \tige{4}{0}{0}       %<- here no dot: tens of thousands
   \tige{5}{3}{0}       %<- here a dot: units thousands
   \tige{6}{0}{1}       %% milhar
   \tige{7}{4}{0}       %% centena
   \tige{8}{0}{0}       %% dezena
   \tige{9}{8}{1}
   \tige{10}{0}{0}
   \tige{11}{0}{0}
   \cadre{11}
\end{tikzpicture}
\end{minipage}%
%
\begin{minipage}{0.4\textwidth}
\begin{tikzpicture}
   \tige{1}{2}{0}
   \tige{2}{5}{0}       %<- here a dot: unit million
   \tige{3}{6}{1}       %<- here no dot: hundreds of thousands
   \tige{4}{0}{0}       %<- here no dot: tens of thousands
   \tige{5}{3}{0}       %<- here a dot: units thousands
   \tige{6}{0}{1}       %% milhar
   \tige{7}{4}{0}       %% centena
   \tige{8}{1}{0}       %% dezena
   \tige{9}{0}{1}
   \tige{10}{0}{0}
   \tige{11}{0}{0}
   \cadre{11}
\end{tikzpicture}
\end{minipage}

\vspace{0.3cm}
\begin{minipage}{0.5\textwidth}
\begin{tikzpicture}
   \tige{1}{2}{0}
   \tige{2}{5}{0}       %<- here a dot: unit million
   \tige{3}{6}{1}       %<- here no dot: hundreds of thousands
   \tige{4}{0}{0}       %<- here no dot: tens of thousands
   \tige{5}{3}{0}       %<- here a dot: units thousands
   \tige{6}{0}{1}       %% milhar
   \tige{7}{4}{0}       %% centena
   \tige{8}{1}{0}       %% dezena
   \tige{9}{0}{1}
   \tige{10}{2}{0}
   \tige{11}{4}{0}
   \cadre{11}
\end{tikzpicture}
\end{minipage}%
%
\begin{minipage}{0.4\textwidth}
\begin{tikzpicture}
   \tige{1}{2}{0}
   \tige{2}{5}{0}       %<- here a dot: unit million
   \tige{3}{6}{1}       %<- here no dot: hundreds of thousands
   \tige{4}{0}{0}       %<- here no dot: tens of thousands
   \tige{5}{3}{0}       %<- here a dot: units thousands
   \tige{6}{0}{1}       %% milhar
   \tige{7}{6}{0}       %% centena
   \tige{8}{1}{0}       %% dezena
   \tige{9}{0}{1}
   \tige{10}{2}{0}
   \tige{11}{4}{0}
   \cadre{11}
\end{tikzpicture}
\end{minipage}


\vspace{0.3cm}
\begin{minipage}{0.5\textwidth}
\begin{tikzpicture}
   \tige{1}{2}{0}
   \tige{2}{5}{0}       %<- here a dot: unit million
   \tige{3}{6}{1}       %<- here no dot: hundreds of thousands
   \tige{4}{0}{0}       %<- here no dot: tens of thousands
   \tige{5}{3}{0}       %<- here a dot: units thousands
   \tige{6}{0}{1}       %% milhar
   \tige{7}{7}{0}       %% centena
   \tige{8}{6}{0}       %% dezena
   \tige{9}{0}{1}
   \tige{10}{2}{0}
   \tige{11}{4}{0}
   \cadre{11}
\end{tikzpicture}
\end{minipage}%
%
\begin{minipage}{0.4\textwidth}
\begin{tikzpicture}
   \tige{1}{2}{0}
   \tige{2}{5}{0}       %<- here a dot: unit million
   \tige{3}{6}{1}       %<- here no dot: hundreds of thousands
   \tige{4}{0}{0}       %<- here no dot: tens of thousands
   \tige{5}{3}{0}       %<- here a dot: units thousands
   \tige{6}{0}{1}       %% milhar
   \tige{7}{7}{0}       %% centena
   \tige{8}{7}{0}       %% dezena
   \tige{9}{8}{1}
   \tige{10}{2}{0}
   \tige{11}{4}{0}
   \cadre{11}
\end{tikzpicture}
\end{minipage}


\vspace{0.35cm}
\paragraph{356 $\times$ 28}= 9968\\
\ladj{0.28}

\begin{minipage}{0.5\textwidth}
\begin{tikzpicture}
   \tige{1}{3}{0}
   \tige{2}{5}{0}       %<- here a dot: unit million
   \tige{3}{6}{1}       %<- here no dot: hundreds of thousands
   \tige{4}{0}{0}       %<- here no dot: tens of thousands
   \tige{5}{0}{0}       %<- here a dot: units thousands
   \tige{6}{2}{1}       %% milhar
   \tige{7}{8}{0}       %% centena
   \tige{8}{2}{0}       %% dezena
   \tige{9}{4}{1}
   \tige{10}{0}{0}
   \tige{11}{0}{0}
   \cadre{11}
\end{tikzpicture}
\end{minipage}%
%
\begin{minipage}{0.4\textwidth}
\begin{tikzpicture}
   \tige{1}{3}{0}
   \tige{2}{5}{0}       %<- here a dot: unit million
   \tige{3}{6}{1}       %<- here no dot: hundreds of thousands
   \tige{4}{0}{0}       %<- here no dot: tens of thousands
   \tige{5}{0}{0}       %<- here a dot: units thousands
   \tige{6}{2}{1}       %% milhar
   \tige{7}{8}{0}       %% centena
   \tige{8}{2}{0}       %% dezena
   \tige{9}{8}{1}
   \tige{10}{0}{0}
   \tige{11}{0}{0}
   \cadre{11}
\end{tikzpicture}
\end{minipage}

\vspace{0.2cm}
\begin{minipage}{0.5\textwidth}
\begin{tikzpicture}
   \tige{1}{3}{0}
   \tige{2}{5}{0}       %<- here a dot: unit million
   \tige{3}{6}{1}       %<- here no dot: hundreds of thousands
   \tige{4}{0}{0}       %<- here no dot: tens of thousands
   \tige{5}{0}{0}       %<- here a dot: units thousands
   \tige{6}{2}{1}       %% milhar
   \tige{7}{0}{0}       %% centena
   \tige{8}{2}{0}       %% dezena
   \tige{9}{8}{1}
   \tige{10}{4}{0}
   \tige{11}{8}{0}
   \cadre{11}
\end{tikzpicture}
\end{minipage}%
%
\begin{minipage}{0.4\textwidth}
\begin{tikzpicture}
   \tige{1}{3}{0}
   \tige{2}{5}{0}       %<- here a dot: unit million
   \tige{3}{6}{1}       %<- here no dot: hundreds of thousands
   \tige{4}{0}{0}       %<- here no dot: tens of thousands
   \tige{5}{0}{0}       %<- here a dot: units thousands
   \tige{6}{2}{1}       %% milhar
   \tige{7}{0}{0}       %% centena
   \tige{8}{8}{0}       %% dezena
   \tige{9}{8}{1}
   \tige{10}{4}{0}
   \tige{11}{8}{0}
   \cadre{11}
\end{tikzpicture}
\end{minipage}


\vspace{0.2cm}
\begin{minipage}{0.5\textwidth}
\begin{tikzpicture}
   \tige{1}{3}{0}
   \tige{2}{5}{0}       %<- here a dot: unit million
   \tige{3}{6}{1}       %<- here no dot: hundreds of thousands
   \tige{4}{0}{0}       %<- here no dot: tens of thousands
   \tige{5}{0}{0}       %<- here a dot: units thousands
   \tige{6}{2}{1}       %% milhar
   \tige{7}{0}{0}       %% centena
   \tige{8}{9}{0}       %% dezena
   \tige{9}{8}{1}
   \tige{10}{4}{0}
   \tige{11}{8}{0}
   \cadre{11}
\end{tikzpicture}
\end{minipage}%
%
\begin{minipage}{0.4\textwidth}
\begin{tikzpicture}
   \tige{1}{3}{0}
   \tige{2}{5}{0}       %<- here a dot: unit million
   \tige{3}{6}{1}       %<- here no dot: hundreds of thousands
   \tige{4}{0}{0}       %<- here no dot: tens of thousands
   \tige{5}{0}{0}       %<- here a dot: units thousands
   \tige{6}{2}{1}       %% milhar
   \tige{7}{0}{0}       %% centena
   \tige{8}{9}{0}       %% dezena
   \tige{9}{9}{1}
   \tige{10}{6}{0}
   \tige{11}{8}{0}
   \cadre{11}
\end{tikzpicture}
\end{minipage}




\vspace{0.35cm}
\paragraph{256 $\times$ 28}= 7168\\
\ladj{0.28}

\begin{minipage}{0.5\textwidth}
\begin{tikzpicture}
   \tige{1}{2}{0}
   \tige{2}{5}{0}       %<- here a dot: unit million
   \tige{3}{6}{1}       %<- here no dot: hundreds of thousands
   \tige{4}{0}{0}       %<- here no dot: tens of thousands
   \tige{5}{0}{0}       %<- here a dot: units thousands
   \tige{6}{2}{1}       %% milhar
   \tige{7}{8}{0}       %% centena
   \tige{8}{1}{0}       %% dezena
   \tige{9}{6}{1}
   \tige{10}{0}{0}
   \tige{11}{0}{0}
   \cadre{11}
\end{tikzpicture}
\end{minipage}%
%
\begin{minipage}{0.4\textwidth}
\begin{tikzpicture}
   \tige{1}{2}{0}
   \tige{2}{5}{0}       %<- here a dot: unit million
   \tige{3}{6}{1}       %<- here no dot: hundreds of thousands
   \tige{4}{0}{0}       %<- here no dot: tens of thousands
   \tige{5}{0}{0}       %<- here a dot: units thousands
   \tige{6}{2}{1}       %% milhar
   \tige{7}{8}{0}       %% centena
   \tige{8}{2}{0}       %% dezena
   \tige{9}{0}{1}
   \tige{10}{0}{0}
   \tige{11}{0}{0}
   \cadre{11}
\end{tikzpicture}
\end{minipage}

\vspace{0.2cm}
\begin{minipage}{0.5\textwidth}
\begin{tikzpicture}
   \tige{1}{3}{0}
   \tige{2}{5}{0}       %<- here a dot: unit million
   \tige{3}{6}{1}       %<- here no dot: hundreds of thousands
   \tige{4}{0}{0}       %<- here no dot: tens of thousands
   \tige{5}{0}{0}       %<- here a dot: units thousands
   \tige{6}{2}{1}       %% milhar
   \tige{7}{0}{0}       %% centena
   \tige{8}{2}{0}       %% dezena
   \tige{9}{0}{1}
   \tige{10}{4}{0}
   \tige{11}{8}{0}
   \cadre{11}
\end{tikzpicture}
\end{minipage}%
%
\begin{minipage}{0.4\textwidth}
\begin{tikzpicture}
   \tige{1}{3}{0}
   \tige{2}{5}{0}       %<- here a dot: unit million
   \tige{3}{6}{1}       %<- here no dot: hundreds of thousands
   \tige{4}{0}{0}       %<- here no dot: tens of thousands
   \tige{5}{0}{0}       %<- here a dot: units thousands
   \tige{6}{2}{1}       %% milhar
   \tige{7}{0}{0}       %% centena
   \tige{8}{6}{0}       %% dezena
   \tige{9}{0}{1}
   \tige{10}{4}{0}
   \tige{11}{8}{0}
   \cadre{11}
\end{tikzpicture}
\end{minipage}


\vspace{0.2cm}
\begin{minipage}{0.5\textwidth}
\begin{tikzpicture}
   \tige{1}{3}{0}
   \tige{2}{5}{0}       %<- here a dot: unit million
   \tige{3}{6}{1}       %<- here no dot: hundreds of thousands
   \tige{4}{0}{0}       %<- here no dot: tens of thousands
   \tige{5}{0}{0}       %<- here a dot: units thousands
   \tige{6}{2}{1}       %% milhar
   \tige{7}{0}{0}       %% centena
   \tige{8}{7}{0}       %% dezena
   \tige{9}{0}{1}
   \tige{10}{4}{0}
   \tige{11}{8}{0}
   \cadre{11}
\end{tikzpicture}
\end{minipage}%
%
\begin{minipage}{0.4\textwidth}
\begin{tikzpicture}
   \tige{1}{3}{0}
   \tige{2}{5}{0}       %<- here a dot: unit million
   \tige{3}{6}{1}       %<- here no dot: hundreds of thousands
   \tige{4}{0}{0}       %<- here no dot: tens of thousands
   \tige{5}{0}{0}       %<- here a dot: units thousands
   \tige{6}{2}{1}       %% milhar
   \tige{7}{0}{0}       %% centena
   \tige{8}{7}{0}       %% dezena
   \tige{9}{1}{1}
   \tige{10}{6}{0}
   \tige{11}{8}{0}
   \cadre{11}
\end{tikzpicture}
\end{minipage}




\vspace{0.35cm}
\paragraph{256 $\times$ 47}= 12032\\
\ladj{0.28}

\begin{minipage}{0.5\textwidth}
\begin{tikzpicture}
   \tige{1}{2}{0}
   \tige{2}{5}{0}       %<- here a dot: unit million
   \tige{3}{6}{1}       %<- here no dot: hundreds of thousands
   \tige{4}{0}{0}       %<- here no dot: tens of thousands
   \tige{5}{0}{0}       %<- here a dot: units thousands
   \tige{6}{4}{1}       %% milhar
   \tige{7}{7}{0}       %% centena
   \tige{8}{1}{0}       %% dezena
   \tige{9}{4}{1}
   \tige{10}{0}{0}
   \tige{11}{0}{0}
   \cadre{11}
\end{tikzpicture}
\end{minipage}%
%
\begin{minipage}{0.4\textwidth}
\begin{tikzpicture}
   \tige{1}{2}{0}
   \tige{2}{5}{0}       %<- here a dot: unit million
   \tige{3}{6}{1}       %<- here no dot: hundreds of thousands
   \tige{4}{0}{0}       %<- here no dot: tens of thousands
   \tige{5}{0}{0}       %<- here a dot: units thousands
   \tige{6}{4}{1}       %% milhar
   \tige{7}{7}{0}       %% centena
   \tige{8}{1}{0}       %% dezena
   \tige{9}{7}{1}
   \tige{10}{5}{0}
   \tige{11}{0}{0}
   \cadre{11}
\end{tikzpicture}
\end{minipage}

\vspace{0.2cm}
\begin{minipage}{0.5\textwidth}
\begin{tikzpicture}
   \tige{1}{2}{0}
   \tige{2}{5}{0}       %<- here a dot: unit million
   \tige{3}{6}{1}       %<- here no dot: hundreds of thousands
   \tige{4}{0}{0}       %<- here no dot: tens of thousands
   \tige{5}{0}{0}       %<- here a dot: units thousands
   \tige{6}{4}{1}       %% milhar
   \tige{7}{0}{0}       %% centena
   \tige{8}{1}{0}       %% dezena
   \tige{9}{7}{1}
   \tige{10}{9}{0}
   \tige{11}{2}{0}
   \cadre{11}
\end{tikzpicture}
\end{minipage}%
%
\begin{minipage}{0.4\textwidth}
\begin{tikzpicture}
   \tige{1}{2}{0}
   \tige{2}{5}{0}       %<- here a dot: unit million
   \tige{3}{6}{1}       %<- here no dot: hundreds of thousands
   \tige{4}{0}{0}       %<- here no dot: tens of thousands
   \tige{5}{0}{0}       %<- here a dot: units thousands
   \tige{6}{4}{1}       %% milhar
   \tige{7}{0}{0}       %% centena
   \tige{8}{9}{0}       %% dezena
   \tige{9}{7}{1}
   \tige{10}{9}{0}
   \tige{11}{2}{0}
   \cadre{11}
\end{tikzpicture}
\end{minipage}


\vspace{0.2cm}
\begin{minipage}{0.5\textwidth}
\begin{tikzpicture}
   \tige{1}{2}{0}
   \tige{2}{5}{0}       %<- here a dot: unit million
   \tige{3}{6}{1}       %<- here no dot: hundreds of thousands
   \tige{4}{0}{0}       %<- here no dot: tens of thousands
   \tige{5}{0}{0}       %<- here a dot: units thousands
   \tige{6}{4}{1}       %% milhar
   \tige{7}{1}{0}       %% centena
   \tige{8}{1}{0}       %% dezena
   \tige{9}{7}{1}
   \tige{10}{9}{0}
   \tige{11}{2}{0}
   \cadre{11}
\end{tikzpicture}
\end{minipage}%
%
\begin{minipage}{0.4\textwidth}
\begin{tikzpicture}
   \tige{1}{3}{0}
   \tige{2}{5}{0}       %<- here a dot: unit million
   \tige{3}{6}{1}       %<- here no dot: hundreds of thousands
   \tige{4}{0}{0}       %<- here no dot: tens of thousands
   \tige{5}{0}{0}       %<- here a dot: units thousands
   \tige{6}{4}{1}       %% milhar
   \tige{7}{1}{0}       %% centena
   \tige{8}{2}{0}       %% dezena
   \tige{9}{0}{1}
   \tige{10}{3}{0}
   \tige{11}{2}{0}
   \cadre{11}
\end{tikzpicture}
\end{minipage}

\chapter{Multiplicação fatorada}
Neste capítulo, vamos aprender uma nova
forma de fazer multiplicações. Suponhamos
que você queira multiplicar 38 por 46.
Coloque o multiplicador 37 (38-1) na
extremidade esquerda
do ábaco e o multiplicando 46 na direita. 
Como já registramos uma ocorrência do
multiplicando, subtraímos 1 do multiplicador.
Lembre-se, nesse método, sempre subtraímos
1 do multiplicador para compensar o registro
do multiplicando no início do processo.

\vspace{0.2cm}
\begin{minipage}{0.5\textwidth}
\begin{tikzpicture}
   \tige{1}{3}{0}
   \tige{2}{7}{0}       %<- here a dot: unit million
   \tige{3}{0}{1}       %<- here no dot: hundreds of thousands
   \tige{4}{0}{0}       %<- here no dot: tens of thousands
   \tige{5}{0}{0}       %<- here a dot: units thousands
   \tige{6}{0}{1}       %% milhar
   \tige{7}{0}{0}       %% centena
   \tige{8}{0}{0}       %% dezena
   \tige{9}{0}{1}
   \tige{10}{4}{0}
   \tige{11}{6}{0}
   \cadre{11}
\end{tikzpicture}
\end{minipage}%
%
\begin{minipage}{0.4\textwidth}
\begin{tikzpicture}
   \tige{1}{3}{0}
   \tige{2}{7}{0}       %<- here a dot: unit million
   \tige{3}{0}{1}       %<- here no dot: hundreds of thousands
   \tige{4}{0}{0}       %<- here no dot: tens of thousands
   \tige{5}{0}{0}       %<- here a dot: units thousands
   \tige{6}{0}{1}       %% milhar
   \tige{7}{0}{0}       %% centena
   \tige{8}{1}{0}       %% dezena
   \tige{9}{2}{1}
   \tige{10}{4}{0}
   \tige{11}{6}{0}
   \cadre{11}
\end{tikzpicture}
\end{minipage}

\vspace{0.2cm}
\begin{minipage}{0.5\textwidth}
\begin{tikzpicture}
   \tige{1}{3}{0}
   \tige{2}{7}{0}       %<- here a dot: unit million
   \tige{3}{0}{1}       %<- here no dot: hundreds of thousands
   \tige{4}{0}{0}       %<- here no dot: tens of thousands
   \tige{5}{0}{0}       %<- here a dot: units thousands
   \tige{6}{0}{1}       %% milhar
   \tige{7}{0}{0}       %% centena
   \tige{8}{1}{0}       %% dezena
   \tige{9}{5}{1}
   \tige{10}{2}{0}
   \tige{11}{6}{0}
   \cadre{11}
\end{tikzpicture}
\end{minipage}%
%
\begin{minipage}{0.4\textwidth}
\begin{tikzpicture}
   \tige{1}{3}{0}
   \tige{2}{7}{0}       %<- here a dot: unit million
   \tige{3}{0}{1}       %<- here no dot: hundreds of thousands
   \tige{4}{0}{0}       %<- here no dot: tens of thousands
   \tige{5}{0}{0}       %<- here a dot: units thousands
   \tige{6}{0}{1}       %% milhar
   \tige{7}{0}{0}       %% centena
   \tige{8}{1}{0}       %% dezena
   \tige{9}{7}{1}
   \tige{10}{0}{0}
   \tige{11}{6}{0}
   \cadre{11}
\end{tikzpicture}
\end{minipage}

\verb||\\
Os passos são:
\begin{enumerate}
\item Registramos $3\times 4= 12$ nas colunas 4 e 3
  a partir da direita. Raciocínio: 3 dezenas (2 dígitos)
  vezes 4 dezenas (mais 2 dígitos) ocupam as
  colunas 4 e 3.
\item Adicionamos $7\times 4= 28$ às colunas 3 e 2.
  Raciocínio: 7 unidades (1 dígito) vezes
  4 dezenas (2 dígitos) vai para a coluna 3 (1+2).
\item Adicionamos $3\times 6= 18$ às colunas 3 e 2.
  Raciocine assim: $3\times 6$ são dezenas (2 dígitos)
  vezes unidades (1 dígito). Um total de três dígitos
  começa na coluna 3.
\end{enumerate}

\vspace{0.2cm}
\begin{minipage}{0.5\textwidth}
\begin{tikzpicture}
   \tige{1}{3}{0}
   \tige{2}{7}{0}       %<- here a dot: unit million
   \tige{3}{0}{1}       %<- here no dot: hundreds of thousands
   \tige{4}{0}{0}       %<- here no dot: tens of thousands
   \tige{5}{0}{0}       %<- here a dot: units thousands
   \tige{6}{0}{1}       %% milhar
   \tige{7}{0}{0}       %% centena
   \tige{8}{1}{0}       %% dezena
   \tige{9}{7}{1}
   \tige{10}{4}{0}
   \tige{11}{8}{0}
   \cadre{11}
\end{tikzpicture}
\end{minipage}%
%
\begin{minipage}{0.4\textwidth}
  Finalmente, adicionamos $7\times 6$
  nas colunas 2 e 1. O resultado final,
  1748 é registrado no lado direito
  do ábaco.
\end{minipage}

\vspace{0.5cm}
\section{Multiplicações continuadas}
A principal desvantagem desse método é
que fica difícil colocar cada par de dígito
da resposta nas colunas corretas.
A grande vantagem desse método é permitir que
se continue com as multiplicações. Por exemplo,
se você quiser multiplicar 1748 por 7, basta
colocar 6 na extremidade esquerda do ábaco
e repetir o processo. O multiplicando 1748
já se encontra na extremidade direita do
ábaco como resultado do produto de 38 por 46.

Vamos recordar novamente
que é preciso subtrair 1 do multiplicador
para levar em conta que o multiplicando
já se encontra posicionado na parte
direita do ábaco. No exemplo, como
queremos multiplicar $7\times 1748$, o multiplicador
é 6, e não 7 como seria de esperar-se.

Lembre-se
de que multiplicação de dígito
por dígito ocupa sempre duas casas.
Assim, $6\times 1$ é 06 e ocupa as
casas 5 e 4 do ábaco, contando da direita
para a esquerda. As etapas do cálculo podem
ser descritas pelos seguintes passos:

\begin{enumerate}
\item Casa 5:  $6\times 1= 06$ --- somar 0
\item Casa 4: $6\times 1= 06$ --- somar 6
\item Casa 4: $6\times 7= 42$ --- somar 40
\item Casa 3: $6\times 7= 42$ --- somar 2
\item Casa 3: $6\times 4= 24$ --- somar 20
\item Casa 2: $6\times 4= 24$ --- somar 4
\item Casa 2: $6\times 8= 48$ --- somar 40
\item Casa 1: $6\times 8= 48$ --- somar 8
\end{enumerate}

Na multiplicação de 6 por 1748,
cada uma dos sucessivos
estados pelos quais o ábaco passa
é indicado na figura abaixo.

\vspace{0.2cm}
\begin{minipage}{0.5\textwidth}
1 --
\begin{tikzpicture}
   \tige{1}{6}{0}
   \tige{2}{0}{0}       %<- here a dot: unit million
   \tige{3}{0}{1}       %<- here no dot: hundreds of thousands
   \tige{4}{0}{0}       %<- here no dot: tens of thousands
   \tige{5}{0}{0}       %<- here a dot: units thousands
   \tige{6}{0}{1}       %% milhar
   \tige{7}{0}{0}       %% centena
   \tige{8}{1}{0}       %% dezena
   \tige{9}{7}{1}
   \tige{10}{4}{0}
   \tige{11}{8}{0}
   \cadre{11}
\end{tikzpicture}
\end{minipage}%
%
\begin{minipage}{0.4\textwidth}
2 -- \begin{tikzpicture}
   \tige{1}{6}{0}
   \tige{2}{0}{0}       %<- here a dot: unit million
   \tige{3}{0}{1}       %<- here no dot: hundreds of thousands
   \tige{4}{0}{0}       %<- here no dot: tens of thousands
   \tige{5}{0}{0}       %<- here a dot: units thousands
   \tige{6}{0}{1}       %% milhar
   \tige{7}{0}{0}       %% centena
   \tige{8}{7}{0}       %% dezena
   \tige{9}{7}{1}
   \tige{10}{4}{0}
   \tige{11}{8}{0}
   \cadre{11}
\end{tikzpicture}
\end{minipage}


\vspace{0.2cm}
\begin{minipage}{0.5\textwidth}
3 -- \begin{tikzpicture}
   \tige{1}{6}{0}
   \tige{2}{0}{0}       %<- here a dot: unit million
   \tige{3}{0}{1}       %<- here no dot: hundreds of thousands
   \tige{4}{0}{0}       %<- here no dot: tens of thousands
   \tige{5}{0}{0}       %<- here a dot: units thousands
   \tige{6}{0}{1}       %% milhar
   \tige{7}{1}{0}       %% centena
   \tige{8}{1}{0}       %% dezena
   \tige{9}{7}{1}
   \tige{10}{4}{0}
   \tige{11}{8}{0}
   \cadre{11}
\end{tikzpicture}
\end{minipage}%
%
\begin{minipage}{0.4\textwidth}
4 -- \begin{tikzpicture}
   \tige{1}{6}{0}
   \tige{2}{0}{0}       %<- here a dot: unit million
   \tige{3}{0}{1}       %<- here no dot: hundreds of thousands
   \tige{4}{0}{0}       %<- here no dot: tens of thousands
   \tige{5}{0}{0}       %<- here a dot: units thousands
   \tige{6}{0}{1}       %% milhar
   \tige{7}{1}{0}       %% centena
   \tige{8}{1}{0}       %% dezena
   \tige{9}{9}{1}
   \tige{10}{4}{0}
   \tige{11}{8}{0}
   \cadre{11}
\end{tikzpicture}
\end{minipage}



\vspace{0.2cm}
\begin{minipage}{0.5\textwidth}
5 -- \begin{tikzpicture}
   \tige{1}{6}{0}
   \tige{2}{0}{0}       %<- here a dot: unit million
   \tige{3}{0}{1}       %<- here no dot: hundreds of thousands
   \tige{4}{0}{0}       %<- here no dot: tens of thousands
   \tige{5}{0}{0}       %<- here a dot: units thousands
   \tige{6}{0}{1}       %% milhar
   \tige{7}{1}{0}       %% centena
   \tige{8}{2}{0}       %% dezena
   \tige{9}{1}{1}
   \tige{10}{4}{0}
   \tige{11}{8}{0}
   \cadre{11}
\end{tikzpicture}
\end{minipage}%
%
\begin{minipage}{0.4\textwidth}
6 -- \begin{tikzpicture}
   \tige{1}{6}{0}
   \tige{2}{0}{0}       %<- here a dot: unit million
   \tige{3}{0}{1}       %<- here no dot: hundreds of thousands
   \tige{4}{0}{0}       %<- here no dot: tens of thousands
   \tige{5}{0}{0}       %<- here a dot: units thousands
   \tige{6}{0}{1}       %% milhar
   \tige{7}{1}{0}       %% centena
   \tige{8}{2}{0}       %% dezena
   \tige{9}{1}{1}
   \tige{10}{8}{0}
   \tige{11}{8}{0}
   \cadre{11}
\end{tikzpicture}
\end{minipage}


\vspace{0.2cm}
\begin{minipage}{0.5\textwidth}
7 -- \begin{tikzpicture}
   \tige{1}{6}{0}
   \tige{2}{0}{0}       %<- here a dot: unit million
   \tige{3}{0}{1}       %<- here no dot: hundreds of thousands
   \tige{4}{0}{0}       %<- here no dot: tens of thousands
   \tige{5}{0}{0}       %<- here a dot: units thousands
   \tige{6}{0}{1}       %% milhar
   \tige{7}{1}{0}       %% centena
   \tige{8}{2}{0}       %% dezena
   \tige{9}{2}{1}
   \tige{10}{2}{0}
   \tige{11}{8}{0}
   \cadre{11}
\end{tikzpicture}
\end{minipage}%
%
\begin{minipage}{0.4\textwidth}
8 -- \begin{tikzpicture}
   \tige{1}{6}{0}
   \tige{2}{0}{0}       %<- here a dot: unit million
   \tige{3}{0}{1}       %<- here no dot: hundreds of thousands
   \tige{4}{0}{0}       %<- here no dot: tens of thousands
   \tige{5}{0}{0}       %<- here a dot: units thousands
   \tige{6}{0}{1}       %% milhar
   \tige{7}{1}{0}       %% centena
   \tige{8}{2}{0}       %% dezena
   \tige{9}{2}{1}
   \tige{10}{3}{0}
   \tige{11}{6}{0}
   \cadre{11}
\end{tikzpicture}
\end{minipage}


\end{document}


